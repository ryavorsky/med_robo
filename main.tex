\documentclass[conference]{IEEEtran}
\IEEEoverridecommandlockouts
\usepackage{cite}
\usepackage{amsmath,amssymb,amsfonts}
\usepackage{algorithmic}
\usepackage{graphicx}
\usepackage{textcomp}
\usepackage{xcolor}
\pagestyle{plain}

%Russian-specific packages
\usepackage[T2A]{fontenc}
\usepackage[utf8]{inputenc}
\usepackage[russian]{babel}

\begin{document}

\title{Обзор публикаций по медицинской робототехнике за период 2019-2021 гг}

\author{\IEEEauthorblockN{Mark Eremenko}
\IEEEauthorblockA{\textit{School of Computer Science and Robotics} \\
\textit{Tomsk Polytechnic University}\\
Tomsk, Russia \\
eremenko@tpu.ru}
\and
\IEEEauthorblockN{Rostislav Yavorskiy}
\IEEEauthorblockA{\textit{School of Computer Science and Robotics} \\
\textit{Tomsk Polytechnic University}\\
Tomsk, Russia \\
ryavorsky@tpu.ru}
}

\maketitle

\begin{abstract}
Представленный обзор публикаций по медицинской робототехнике за период 2019-2021 гг. аггрегирует информацию о более чем 150 наиболее релевантных публикациях по этой тематике в англоязычной научной литературе.
\end{abstract}

\begin{IEEEkeywords}
medical robotics, healthcare robotics, remote surgery 
\end{IEEEkeywords}

\medskip
\subsubsection{Erin, Onder, Mustafa Boyvat, Mehmet Efe Tiryaki, Martin Phelan, and Metin Sitti. "Magnetic resonance imaging system–driven medical robotics." Advanced Intelligent Systems 2, no. 2 (2020): 1900110.}
see \cite{erin2020magnetic}

Magnetic resonance imaging (MRI) system–driven medical robotics is an emerging field that aims to use clinical MRI systems not only for medical imaging but also for actuation, localization, and control of medical robots. Submillimeter scale resolution of MR images for soft tissues combined with the electromagnetic gradient coil–based magnetic actuation available inside MR scanners can enable theranostic applications of medical robots for precise image-guided minimally invasive interventions. MRI-driven robotics typically does not introduce new MRI instrumentation for actuation but instead focuses on converting already available instrumentation for robotic purposes. To use the advantages of this technology, various medical devices such as untethered mobile magnetic robots and tethered active catheters have been designed to be powered magnetically inside MRI systems. Herein, the state-of-the-art progress, challenges, and future directions of MRI-driven medical robotic systems are reviewed.


\medskip
\subsubsection{Strydom, Mario, Artur Banach, Liao Wu, Anjali Jaiprakash, Ross Crawford, and Jonathan Roberts. "Anatomical joint measurement with application to medical robotics." IEEE Access 8 (2020): 118510-118524.}
see \cite{strydom2020anatomical}

Robotic-assisted orthopaedic procedures demand accurate spatial joint measurements. Tracking of human joint motion is challenging in many applications, such as in sport motion analyses. In orthopaedic surgery, these challenges are even more prevalent, where small errors may cause iatrogenic damage in patients - highlighting the need for robust and precise joint and instrument tracking methods. In this study, we present a novel kinematic modelling approach to track any anatomical points on the femur and / or tibia by exploiting optical tracking measurements combined with a priori computed tomography information. The framework supports simultaneous tracking of anatomical positions, from which we calculate the pose of the leg (joint angles and translations of both the hip and knee joints) and of each of the surgical instruments. Experimental validation on cadaveric data shows that our method is capable of measuring these anatomical regions with sub-millimetre accuracy, with a maximum joint angle uncertainty of ±0.47°. This study is a fundamental step in robotic orthopaedic research, which can be used as a ground-truth for future research such as automating leg manipulation in orthopaedic procedures.

\medskip
\subsubsection{von Haxthausen, Felix, Sven Böttger, Daniel Wulff, Jannis Hagenah, Verónica García-Vázquez, and Svenja Ipsen. "Medical robotics for ultrasound imaging: current systems and future trends." Current Robotics Reports (2021): 1-17.}
see \cite{von2021medical}

Purpose of Review
This review provides an overview of the most recent robotic ultrasound systems that have contemporary emerged over the past five years, highlighting their status and future directions. The systems are categorized based on their level of robot autonomy (LORA).

Recent Findings
Teleoperating systems show the highest level of technical maturity. Collaborative assisting and autonomous systems are still in the research phase, with a focus on ultrasound image processing and force adaptation strategies. However, missing key factors are clinical studies and appropriate safety strategies. Future research will likely focus on artificial intelligence and virtual/augmented reality to improve image understanding and ergonomics.

Summary
A review on robotic ultrasound systems is presented in which first technical specifications are outlined. Hereafter, the literature of the past five years is subdivided into teleoperation, collaborative assistance, or autonomous systems based on LORA. Finally, future trends for robotic ultrasound systems are reviewed with a focus on artificial intelligence and virtual/augmented reality.

\medskip
\subsubsection{Li, Kun, and Joel W. Burdick. "Human motion analysis in medical robotics via high-dimensional inverse reinforcement learning." The International Journal of Robotics Research 39, no. 5 (2020): 568-585.}
see \cite{li2020human}

This work develops a novel high-dimensional inverse reinforcement learning (IRL) algorithm for human motion analysis in medical, clinical, and robotics applications. The method is based on the assumption that a surgical robot operators’ skill or a patient’s motor skill is encoded into the innate reward function during motion planning and recovered by an IRL algorithm from motion demonstrations. This class of applications is characterized by high-dimensional sensory data, which is computationally prohibitive for most existing IRL algorithms. We propose a novel function approximation framework and reformulate the Bellman optimality equation to handle high-dimensional state spaces efficiently. We compare different function approximators in simulated environments, and adopt a deep neural network as the function approximator. The technique is applied to evaluating human patients with spinal cord injuries under spinal stimulation, and the skill levels of surgical robot operators. The results demonstrate the efficiency and effectiveness of the proposed method.

\medskip
\subsubsection{Fotouhi, Javad, Tianyu Song, Arian Mehrfard, Giacomo Taylor, Qiaochu Wang, Fengfan Xian, Alejandro Martin-Gomez et al. "Reflective-ar display: An interaction methodology for virtual-to-real alignment in medical robotics." IEEE Robotics and Automation Letters 5, no. 2 (2020): 2722-2729.}
see \cite{fotouhi2020reflective}

Robot-assisted minimally invasive surgery has shown to improve patient outcomes, as well as reduce complications and recovery time for several clinical applications. While increasingly configurable robotic arms can maximize reach and avoid collisions in cluttered environments, positioning them appropriately during surgery is complicated because safety regulations prevent automatic driving. We propose a head-mounted display (HMD) based augmented reality (AR) system designed to guide optimal surgical arm set up. The staff equipped with HMD aligns the robot with its planned virtual counterpart. In this user-centric setting, the main challenge is the perspective ambiguities hindering such collaborative robotic solution. To overcome this challenge, we introduce a novel registration concept for intuitive alignment of AR content to its physical counterpart by providing a multi-view AR experience via reflective-AR displays that simultaneously show the augmentations from multiple viewpoints. Using this system, users can visualize different perspectives while actively adjusting the pose to determine the registration transformation that most closely superimposes the virtual onto the real. The experimental results demonstrate improvement in the interactive alignment a virtual and real robot when using a reflective-AR display. We also present measurements from configuring a robotic manipulator in a simulated trocar placement surgery using the AR guidance methodology.


\medskip
\subsubsection{Ozmen, M. Mahir, Asutay Ozmen, and Çetin Kaya Koç. "Artificial Intelligence for Next-Generation Medical Robotics." In Digital Surgery, pp. 25-36. Springer, Cham, 2021.}
see \cite{ozmen2021artificial}

Technology-based advancements have the potential to empower every surgeon with the ability to improve the quality of global surgical care. Innovation in robotic surgery will continue to parallel advancements in technology, especially with the considerable progress in computer science and artificial intelligence (AI). It is also known that high-quality surgical techniques and skill sets correlate positively with patient outcomes. AI could help pool this surgical experience to standardize decision-making, thus creating a global consensus in operating theaters worldwide. Next-generation surgical robots will be integral in augmenting a surgeon’s skills effectively to achieve accuracy and high precision during complex procedures. The next level of surgery will be achieved by surgical robotics which likely evolve to include AI and machine learning.

\medskip
\subsubsection{Zheng, Jia, Shuangyi Wang, James Housden, Zeng-Guang Hou, Davinder Singh, and Kawal Rhode. "A Safety Joint with Passive Compliant and Manual Override Mechanisms for Medical Robotics." In 2021 IEEE International Conference on Intelligence and Safety for Robotics (ISR), pp. 1-4. IEEE, 2021.}
see \cite{zheng2021safety}

Force and collision control is a primary concern to guarantee the safe use of medical robots as such systems normally need to interact with clinicians and patients, while at the same time cooperate with other devices. Among different strategies, passive features working with intrinsically safety components are treated as one of the most effective approaches and therefore deserve in-depth study. In this study, we focus on the design of a novel back-drivable safety joint that incorporates a torque limiter with passive compliance and a manual override mechanism to disconnect the robotic joint from its drive train. The design and working principle of the proposed joint are explained, followed by the mathematical analysis of its performance and the impacts of parameters. An example of the design was manufactured and tested experimentally to validate the working concepts. It is concluded that the proposed multi-functional safety joint provides more versatility and customization to the design of bespoke medical robots and would limit the maximum torque that can be exserted onto the patient, allow the clinician to push the joint back, and enable the operator to switch back to manual override.


\medskip
\subsubsection{Starszak, Krzysztof, Michał Smoczok, and Weronika Starszak. "New technologies in health care—medical robotics and innovations during the COVID-19 pandemic, considering Polish achievements." Chirurgia Polska (2021).}
see \cite{starszak2021new}

In March 2020, the WHO declared a state of a pandemic, which encompassed the whole world. During
the pandemic, numerous new solutions have been introduced and some of the already existing ones
have been improved to increase the safety, both of patients and healthcare professionals. The publication
aims to present the achievements in the field of innovations with the effects of the Covid-19 pandemic,
considering the activities of Polish scientists. The literature and current data were reviewed and useful in
the topic of research were selected. The pandemic period showed the interdisciplinary nature of medical
robots, both for surgical and diagnostic purposes. Robots are widely used in cleaning and disinfecting
rooms. Patient psychological care systems also deserve attention - during the pandemic, the number of
those in need suffering from mental diseases increased. Medical robotics should be developed and used
more and more commonly.

\medskip
\subsubsection{Makhataeva, Zhanat, and Huseyin Atakan Varol. "Augmented reality for robotics: a review." Robotics 9, no. 2 (2020): 21.}
see \cite{makhataeva2020augmented}

Augmented reality (AR) is used to enhance the perception of the real world by integrating virtual objects to an image sequence acquired from various camera technologies. Numerous AR applications in robotics have been developed in recent years. The aim of this paper is to provide an overview of AR research in robotics during the five year period from 2015 to 2019. We classified these works in terms of application areas into four categories: (1) Medical robotics: Robot-Assisted surgery (RAS), prosthetics, rehabilitation, and training systems; (2) Motion planning and control: trajectory generation, robot programming, simulation, and manipulation; (3) Human-robot interaction (HRI): teleoperation, collaborative interfaces, wearable robots, haptic interfaces, brain-computer interfaces (BCIs), and gaming; (4) Multi-agent systems: use of visual feedback to remotely control drones, robot swarms, and robots with shared workspace. Recent developments in AR technology are discussed followed by the challenges met in AR due to issues of camera localization, environment mapping, and registration. We explore AR applications in terms of how AR was integrated and which improvements it introduced to corresponding fields of robotics. In addition, we summarize the major limitations of the presented applications in each category. Finally, we conclude our review with future directions of AR research in robotics. The survey covers over 100 research works published over the last five years.

\medskip
\subsubsection{Scimeca, Luca, Fumiya Iida, Perla Maiolino, and Thrishantha Nanayakkara. "Human-Robot Medical Interaction." In Companion of the 2020 ACM/IEEE International Conference on Human-Robot Interaction, pp. 660-661. 2020.}
see \cite{scimeca2020human}

Advances in Soft Robotics, Haptics, AI and simulation have changed the medical robotics field, allowing robotics technologies to be deployed in medical environments. In this context, the relationship between doctors, robotics devices, and patients is fundamental, as only with the synergetic collaboration of the three parties results in medical robotics can be achieved. This workshop focuses on the use of soft robotics technologies, sensing, AI and Simulation, to further improve medical practitioner training, as well as the creation of new tools for diagnosis and healthcare through the medical interaction of humans and robots. The Robo-patient is more specifically the idea behind the creation of sensorised robotic patient with controllable organs to present a given set of physiological conditions. This is both to investigate the embodied nature of haptic interaction in physical examination, as well as the doctor-patient relationship to further improve medical practice through robotics technologies. The Robo-doctor aspect is also relevant, with robotics prototypes performing, or helping to perform, medical diagnosis. In the workshop, key technologies as well as future views in the field will be discussed both by expert and new upcoming researchers.

\medskip
\subsubsection{DCRUST, Murthal, Jasbir Singh Saini, and Sanjeev Kumar. "Internet of Medical Things." Patron in Chief: 145.}
see \cite{dcrustinternet}

Internet of Medical Things (IoMT) enables machine to machine interaction, real time
intervention, better affordability and more reliability in future of healthcare. IoMT has
applications in chronic disease management, tele-health, lifestyle assessment, remote
intervention, drug management, medical robotics, etc. The various technologies used for
implementing IoMT involve the macro-level IoT architecture comprising layers to enable
healthcare solutions. These smart medical devices when connected to other medical devices
allow patient management, surgical intervention, etc. The key players in the market of smart
medical things dominate the Intellectual field worldwide. Keeping in view the benefits and
related challenges, IoMT appears to be a valuable solution to benefit healthcare monitoring,
diagnosis and treatment procedures. We have incorporated the analysis of such applications in
this paper.

\medskip
\subsubsection{Goel, Rahul. "Role of Robotics in Health Care of the Future." Journal of Medical Academics 3, no. 1 (2020): 28.}
see \cite{goel2020role}

As the demands on medical professionals and healthcare infrastructure increase, the introduction of automation via robotics is inevitable.
Robotics originated in science fiction literature and from there industrial robotic arms, and more recently surgical robotic devices have been
created. In this article, we examine the types of robots, their development, and upcoming projects.

\medskip
\subsubsection{Dixit, Pooja, Manju Payal, Nidhi Goyal, and Vishal Dutt. "Robotics, AI and IoT in Medical and Healthcare Applications." AI and IoT‐Based Intelligent Automation in Robotics (2021): 53-73.}
see \cite{dixit2021robotics}

Today, the vital role of robotics, AI and IOT technologies have recast healthcare. Healthcare apps enabled by these technologies help manage the health of consumers, thus ensuring their health. The main focus of this chapter is to study the applications for the techniques that make the healthcare system more affordable, provide better outcomes and also access patients' records in order to provide better solutions. Thus, when these technologies merge, there is a chance that they will be capable of better operational efficiency for tracking and monitoring patients, with automation making more optimistic solutions possible.

\medskip
\subsubsection{Niu, Guojun, Bo Pan, Yili Fu, and Cuicui Qu. "Development of a new medical robot system for minimally invasive surgery." IEEE Access 8 (2020): 144136-144155.}
see \cite{niu2020development}

This article presents the development of a new medical robot system comprising a spherical remote center motion (RCM) mechanism with modular design and two mechanical decoupling methods for Minimally Invasive Surgery (MIS). We achieved excellent comprehensive performance indices through a novel multi-objective optimization model comprising four optimization objective functions, three constrained conditions and two optimization variables. In order to enhance the manipulability, remove the coupling between motors, and reduce the control difficulty, two new decoupling mechanism means were proposed to remove coupling motion between the wrist and pincers, coupling motion between the translational joint of mobile platform and four interface disks of surgical instrument as a results of rear drive motor, respectively. The control system architecture is designed to include intuitive motion control, incremental motion control, and proportional motion control. Master-slave attitude registration and surgical instrument replacement strategies improve the master-slave control efficiency. We tested the spherical RCM mechanism performance indices and developed two mechanical decoupling methods and a master-slave control algorithm. Our experimental test results validated that fixing point accuracy, the coupling motions, the positioning and repeated positioning accuracy of the MIS robot, and master-slave control algorithm meet the requirements of MIS. Successful animal experiments confirmed effectiveness of the novel MIS robot system.

\medskip
\subsubsection{Begishev, Ildar, Zarina Khisamova, and Vitaly Vasyukov. "Technological, Ethical, Environmental and Legal Aspects of Robotics." In E3S Web of Conferences, vol. 244, p. 12028. EDP Sciences, 2021.}
see \cite{begishev2021technological}

Robotics is considered by modern researchers from various positions. The most common technical approach to the study of this concept, which examines the current state and achievements in the field of robotics, as well as the prospects for its development. Also, quite often in recent years, legal experts have begun to address problems related to the development of robotics, focusing on issues related to the legal personality of robots and artificial intelligence, as well as the responsibility of AI for causing harm. A separate direction in the field of robotics research is the analysis of this concept and the relations associated with it, from the standpoint of morality, ethics and technologies.

\medskip
\subsubsection{Boiadjiev, Tony, George Boiadjiev, Kamen Delchev, Ivan Chavdarov, and Roumen Kastelov. "Orthopedic Bone Drilling Robot ODRO: Basic Characteristics and Areas of Applications." In Medical Robotics. IntechOpen, 2021.}
see \cite{boiadjiev2021orthopedic}

The orthopedic manipulation “bone drilling” is the most executed one in the orthopedic surgery concerning the operative treatment of bone fractures. The drilling process is characterized by a number of input and output parameters. The most important input parameters are the feed rate [mm/s] and the drill speed [rpm]. They play significant role for the final result (the output parameters): thermal and mechanical damages of the bone tissue as well as hole quality. During the manual drilling these parameters are controlled by the surgeon on the base of his practical skills. But the optimal results of the manipulations can be assured only when the input parameters are under control during an automatic execution of the drilling process. This work presents the functional characteristics of the handheld robotized system ODRO (Orthopedic Drilling Robot) for automatic bone drilling. Some experimental results are also shown. A comparison is made between the similar systems which are known in the literature, some of which are available on the market. The application areas of ODRO in the orthopedic surgery practice are underlined.

\medskip
\subsubsection{Ginoya, Tirth, Yaser Maddahi, and Kourosh Zareinia. "A historical review of medical robotic platforms." Journal of Robotics 2021 (2021).}
see \cite{ginoya2021historical}

This paper provides a brief history of medical robotic systems. Since the first use of robots in medical procedures, there have been countless companies competing to developed robotic systems in hopes to dominate a field. Many companies have succeeded, and many have failed. This review paper shows the timeline history of some of the old and most successful medical robots and new robotic systems. As the patents of the most successful system, i.e., Da Vinci® Surgical System, have expired or are expiring soon, this paper can provide some insights for new designers and manufacturers to explore new opportunities in this field.

\medskip
\subsubsection{Gruionu, Lucian Gheorghe, Catalin Constantinescu, Andreea Iacob, and Gabriel Gruionu. "Robotic System for Catheter Navigation during Medical Procedures." In Applied Mechanics and Materials, vol. 896, pp. 211-217. Trans Tech Publications Ltd, 2020.}
see \cite{gruionu2020robotic}

ung cancer is the most common cancer globally with over 2 million new cases diagnosed every year. Fortunately, if caught early, the likelihood of survival is greatly improved. If diagnosed in Stage I, survival rates are > 75 \% over 5 years, vs. just 1 \%  if diagnosed in Stage IV. Early diagnosis requires finding and sampling (biopsy) small, peripheral nodules that are located in the parenchima of the lung and predominately outside small airways. Currently, for early diagnosis a bronchoscope is inserted into the lung airway but due to large size it cannot reach the small airways. Therefore, the doctor has to advance a sharp biopsy needle blindly from the tip of the bronchoscope and into the lung tissue in the approximate direction of the nodule. This blind procedure has low accuracy and carries a high risk of misdiagnosis. Currently, to improve the accuracy, real time x-ray (fluoroscopy) is use which causes exposure of the patient and physician to harmful radiation. Computer and image assisted surgery and medical robotics present viable solutions but are not optimal at present. The scope of our research was to develop a robotic solution for increased precision and accuracy of early diagnosis and treatment of lung cancer, to increase procedure success rate, decrease patient radiation and stress exposure, and reduce the procedure cost. For this purpose, we developed an advanced prototype of a robotic system which is small in size, easy to use and effective. To demonstrate its effectiveness in navigating to peripheral small size lung cancer lesions, we performed laboratory tests or a realistic lung airway model. The preliminary tests of a novel medical robot using a complex lung airway model proved that our catheter driving robotic system is working as designed and allows navigation, through a complex 3D channels structure like the bronchial tree, in both manipulator and robotic modes without fluoroscopy scanning. The robotic system is more precise and stable, and can avoid patient injury and instrument damage due to accidental impact with the airway wall. Because it could be controlled from a different room via the software platform, using this robotic system can drastically reduce radiation exposure of the patient and totally avoid the exposure of the doctor. Another benefit of the proposed robotic system is that it uses currently available catheters in which a reusable electromagnetic guide wire is temporarily inserted to guide the tip of the catheter towards hard to reach targets. After the target is confirmed, the sensor can be retracted and the catheter can be used for its routine function such as biopsy collection. Future development will include placement of a force sensor at the tip of the catheter to “feel” the wall and adapt the speed of insertion in order to avoid wall damage and an improved algorithm to increase the speed in the automatic mode.

\medskip
\subsubsection{Tian, Xiumei, and Yan Xu. "Low Delay Control Algorithm of Robot Arm for Minimally Invasive Medical Surgery." IEEE Access 8 (2020): 93548-93560.}
see \cite{tian2020low}

Minimally invasive surgical robots have received more and more attention from medical patients because of their higher surgical accuracy and higher safety than doctors. Minimally invasive surgery is rapidly revolutionizing the treatment of traditional surgery. In order to solve the problem that the surgical robot has a redundant degree of freedom, which makes the kinematics solution more complicated, this paper analyzes the kinematics of the coordinate system block. Aiming at the problem that the strategy search algorithm needs to re-learn when the target pose changes, a convolutional neural network control strategy is studied and constructed. By designing the structure of the convolutional neural network visual layer and motor control layer, the loss function and sampling of the training process are established. Aiming at the problem of long training time of convolutional neural network, an effective pre-training method is proposed to shorten the training time of the neural network. At the same time, the effectiveness of the above method and the end-to-end control of the convolutional neural network strategy are verified through simulation experiments. The physical structure of the manipulator body is analyzed, and the forward and inverse kinematic equations of the manipulator are established by the D-H method. Monte Carlo method was used to analyze the working space of the manipulator, and low-latency control and simulation experiments were carried out on the movement trajectory of the manipulator in joint space and Cartesian space. The results show that the low-latency control algorithm in this paper is effective to control the mechanical arm of the minimally invasive medical surgery robot.

\medskip
\subsubsection{Kennedy-Metz, Lauren R., Pietro Mascagni, Antonio Torralba, Roger D. Dias, Pietro Perona, Julie A. Shah, Nicolas Padoy, and Marco A. Zenati. "Computer Vision in the Operating Room: Opportunities and Caveats." IEEE transactions on medical robotics and bionics 3, no. 1 (2020): 2-10.}
see \cite{kennedy2020computer}

Effectiveness of computer vision techniques has been demonstrated through a number of applications, both within and outside healthcare. The operating room environment specifically is a setting with rich data sources compatible with computational approaches and high potential for direct patient benefit. The aim of this review is to summarize major topics in computer vision for surgical domains. The major capabilities of computer vision are described as an aid to surgical teams to improve performance and contribute to enhanced patient safety. Literature was identified through leading experts in the fields of surgery, computational analysis and modeling in medicine, and computer vision in healthcare. The literature supports the application of computer vision principles to surgery. Potential applications within surgery include operating room vigilance, endoscopic vigilance, and individual and team-wide behavioral analysis. To advance the field, we recommend collecting and publishing carefully annotated datasets. Doing so will enable the surgery community to collectively define well-specified common objectives for automated systems, spur academic research, mobilize industry, and provide benchmarks with which we can track progress. Leveraging computer vision approaches through interdisciplinary collaboration and advanced approaches to data acquisition, modeling, interpretation, and integration promises a powerful impact on patient safety, public health, and financial costs.

\medskip
\subsubsection{Maibaum, Arne, Andreas Bischof, Jannis Hergesell, and Benjamin Lipp. "A critique of robotics in health care." AI \& SOCIETY (2021): 1-11.}
see \cite{maibaum2021critique}

When the social relevance of robotic applications is addressed today, the use of assistive technology in care settings is almost always the first example. So-called care robots are presented as a solution to the nursing crisis, despite doubts about their technological readiness and the lack of concrete usage scenarios in everyday nursing practice. We inquire into this interconnection of social robotics and care. We show how both are made available for each other in three arenas: innovation policy, care organization, and robotic engineering. First, we analyze the discursive “logics” of care robotics within European innovation policy, second, we disclose how care robotics is encountering a historically grown conflict within health care organization, and third we show how care scenarios are being used in robotic engineering. From this diagnosis, we derive a threefold critique of robotics in healthcare, which calls attention to the politics, historicity, and social situatedness of care robotics in elderly care.

\medskip
\subsubsection{Kyrarini, Maria, Fotios Lygerakis, Akilesh Rajavenkatanarayanan, Christos Sevastopoulos, Harish Ram Nambiappan, Kodur Krishna Chaitanya, Ashwin Ramesh Babu, Joanne Mathew, and Fillia Makedon. "A survey of robots in healthcare." Technologies 9, no. 1 (2021): 8.}
see \cite{kyrarini2021survey}

In recent years, with the current advancements in Robotics and Artificial Intelligence (AI), robots have the potential to support the field of healthcare. Robotic systems are often introduced in the care of the elderly, children, and persons with disabilities, in hospitals, in rehabilitation and walking assistance, and other healthcare situations. In this survey paper, the recent advances in robotic technology applied in the healthcare domain are discussed. The paper provides detailed information about state-of-the-art research in care, hospital, assistive, rehabilitation, and walking assisting robots. The paper also discusses the open challenges healthcare robots face to be integrated into our society.

\medskip
\subsubsection{Tsigie, Sisay Ebabye, and Gizealew Alazie Dagnaw. "The Role of Robotics Technology and Internet of Things for Industry 4.0 Realization." International Journal 10, no. 2 (2021).}
see \cite{tsigie2021role}

Robotic systems can already proactively monitor and adapt to changes in a production line. Nowadays, internet of things and robotic systems are key drivers of technological innovation trends.Major companies are now making investments in machine learning-powered approaches to improve in principle all aspects of manufacturing. Connected devices, sensors, and similar advancements allow people and companies to do things they wouldn't even dream of in earlier eras. For realizing it time series feature extraction approach is selected.Industrial internet of things solutions are poised to transform many industry verticals including healthcare, retail, automotive, and transport. For many industries, the industrial internet of things has significantly improved reliability, production, and customer satisfaction. The internet of things and robotics are coming together to create the internet of robotic things. Industrial internet of thing is a subset of industry 4.0. It can encourage smartness at a bigger level in industrial robots.

\medskip
\subsubsection{Daghottra, Ankita, and Divya Jain. "From Humans to Robots: Machine Learning for Healthcare." (2021).}
see \cite{daghottra2021humans}

Machine learning is a branch of artificial intelligence (AI) through which identification of patterns in data is done and with help of these patterns, useful outcomes or conclusions are predicted. One of the most prominent or frequently studied applications of machine learning is the surgical phase or robotic surgery. This makes machine learning an important part of research in robotics. The implementation of this technology in the field of healthcare aims in improving medical practices resulting in more precise and advanced surgical assessments. This paper aims in outlining the implementation and applications of machine learning related to robotics in the field of healthcare. Machine learning aims in generating positive outcomes with assumptions. The objective of this paper is to bring light on how these technologies have become an important part of providing more effective and comprehensive strategies which eventually add to positive patient outcomes and more advanced healthcare practices.

\medskip
\subsubsection{Porkodi, S., and D. Kesavaraja. "Healthcare Robots Enabled with IoT and Artificial Intelligence for Elderly Patients." AI and IoT‐Based Intelligent Automation in Robotics (2021): 87-108.}
see \cite{porkodi2021healthcare}

As the demand for doctors is increasing day by day, a need has arisen to provide personalized healthcare for elderly patients and those with chronic conditions in addition to taking necessary actions during emergency situations. So, healthcare in the digital era is experimenting with adopting robotics to provide personal healthcare to patients in need. In this chapter, the needs of elderly patients are identified and solutions are provided with a personalized robot. Emergency situations can also be predicted more quickly with the vital information provided by IoT devices and necessary action can be suggested using artificial intelligence. IoT-based wearable devices are used to obtain necessary health data from patients. These data are processed and decision-making is carried out by AI, whereas the required action is taken by the designed robot. Humanoid robots can be designed for providing healthcare and physical assistance to elderly patients and those with chronic conditions. Animal-like robots can also be designed that act like pets as a solution for those with psychosocial issues. The major goal is to review robots to develop a robot in the future that can prevent interventions, perform multiple functions, provide motivational interaction style, provide better educational data, and alert an ambulance in case of an emergency.

\medskip
\subsubsection{Luo, Shi, Xi Zhou, Xinyue Tang, Jialu Li, Dacheng Wei, Guojun Tai, Zongyong Chen et al. "Microconformal electrode-dielectric integration for flexible ultrasensitive robotic tactile sensing." Nano Energy 80 (2021): 105580.}
see \cite{luo2021microconformal}

Flexible pressure sensors have attracted a lot of interest because of their widespread applications in healthcare, robotics, wearable smart devices, and human-machine interfaces. While microstructuring both the electrodes and dielectrics has been proven to have a significant improvement in the sensitivity and response speed of piezocapacitive sensors, the synergetic influence of microstructured electrodes and dielectrics has not been discussed yet. Herein, a flexible piezocapacitive sensor has been demonstrated with a microstructured graphene nanowalls (GNWs) electrode and a conformally microstructured dielectric layer that consists of polydimethylsiloxane (PDMS) and piezoelectric enhancer of zinc oxide (ZnO). Such microstructured assembly with piezoelectric film constructs a microconformal GNWs/PDMS/ZnO electrode-dielectric integration (MEDI), which can effectively enhance the sensitivity and the pressure-response range. The piezocapacitive sensor exhibits an ultra-high sensitivity (22.3 kPa-1), fast response speed (25 ms), and broad pressure range (22 kPa). The finite element analysis indicates that the polarized electric field caused by the ZnO film’s piezoelectric effect greatly enhances the capacitance of the sensor. Moreover, the integration of the electrode and dielectric layer can eliminate the slippage between contiguous layers, which effectively increases the mechanical stability. Benefitting from the outstanding comprehensive performance, the potential application in robotic tactile perception has been successfully demonstrated, including object grabbing, braille recognition, and roughness detection. The MEDI in structure capacitive sensors provides a new approach to achieve high-performance E-skin, which delivers great potential applications in nextgeneration robotic tactile sensing.

\medskip
\subsubsection{Lestingi, Livia, Mehrnoosh Askarpour, Marcello M. Bersani, and Matteo Rossi. "Formal verification of human-robot interaction in healthcare scenarios." In International Conference on Software Engineering and Formal Methods, pp. 303-324. Springer, Cham, 2020.}
see \cite{lestingi2020formal}

We present a model-driven approach for the creation of formally verified scenarios involving human-robot interaction in healthcare settings. The work offers an innovative take on the application of formal methods to human modeling, as it incorporates physiology-related aspects. The model, based on the formalism of Hybrid Automata, includes a stochastic component to capture the variability of human behavior, which makes it suitable for Statistical Model Checking. The toolchain is meant to be accessible to a wide range of professional figures. Therefore, we have laid out a user-friendly representation format for the scenario, from which the full formal model is automatically generated and verified through the Uppaal tool. The outcome is an estimation of the probability of success of the mission, based on which the user can refine the model if the result is not satisfactory.

\medskip
\subsubsection{Cifuentes, Carlos A., Maria J. Pinto, Nathalia Céspedes, and Marcela Múnera. "Social robots in therapy and care." Current Robotics Reports (2020): 1-16.}
see \cite{cifuentes2020social}

Purpose of Review

This work presents a comprehensive overview of social robots in therapy and the healthcare of children, adults, and elderly populations. According to recent evidence in this field, the primary outcomes and limitations are highlighted. This review points out the implications and requirements for the proper deployment of social robots in therapy and healthcare scenarios.

Recent Findings

Social robots are a current trend that is being studied in different healthcare services. Evidence highlights the potential and favorable results due to the support and assistance provided by social robots. However, some side effects and limitations are still under research.

Summary

Social robots can play various roles in the area of health and well-being. However, further studies regarding the acceptability and perception are still required. There are challenges to be addressed, such as improvements in the functionality and robustness of these robotic systems.

\medskip
\subsubsection{James, Jesin, B. T. Balamurali, Catherine I. Watson, and Bruce MacDonald. "Empathetic Speech Synthesis and Testing for Healthcare Robots." International Journal of Social Robotics (2020): 1-19.}
see \cite{james2020empathetic}

One of the major factors that affect the acceptance of robots in Human-Robot Interaction applications is the type of voice with which they interact with humans. The robot’s voice can be used to express empathy, which is an affective response of the robot to the human user. In this study, the aim is to find out if social robots with empathetic voice are acceptable for users in healthcare applications. A pilot study using an empathetic voice spoken by a voice actor was conducted. Only prosody in speech is used to express empathy here, without any visual cues. Also, the emotions needed for an empathetic voice are identified. It was found that the emotions needed are not only the stronger primary emotions, but also the nuanced secondary emotions. These emotions are then synthesised using prosody modelling. A second study, replicating the pilot test is conducted using the synthesised voices to investigate if empathy is perceived from the synthetic voice as well. This paper reports the modelling and synthesises of an empathetic voice, and experimentally shows that people prefer empathetic voice for healthcare robots. The results can be further used to develop empathetic social robots, that can improve people’s acceptance of social robots.

\medskip
\subsubsection{Attanasio, Aleks, Bruno Scaglioni, Elena De Momi, Paolo Fiorini, and Pietro Valdastri. "Autonomy in surgical robotics." Annual Review of Control, Robotics, and Autonomous Systems 4 (2021): 651-679.}
see \cite{attanasio2021autonomy}

This review examines the dichotomy between automatic and autonomous behaviors in surgical robots, maps the possible levels of autonomy of these robots, and describes the primary enabling technologies that are driving research in this field. It is organized in five main sections that cover increasing levels of autonomy. At level 0, where the bulk of commercial platforms are, the robot has no decision autonomy. At level 1, the robot can provide cognitive and physical assistance to the surgeon, while at level 2, it can autonomously perform a surgical task. Level 3 comes with conditional autonomy, enabling the robot to plan a task and update planning during execution. Finally, robots at level 4 can plan and execute a sequence of surgical tasks autonomously.

\medskip
\subsubsection{Narejo, Ghous Bakhsh. "Robotics and Machine Learning." In Privacy Vulnerabilities and Data Security Challenges in the IoT, pp. 183-198. CRC Press, 2020.}
see \cite{narejo2020robotics}

Robotics can be described as an interdisciplinary branch of engineering that may include mechanical, electronics, information, computer, and other engineering faculties. The main purpose of robotics is to deal with the design, construction, operation, and use of robots, as well as computer devices for their control, feedback, and the processing of data.

\medskip
\subsubsection{Rahangdale, Swapnil, Dezi Maind, Sakshi Amle, Komal Yadav, Pradip Dhore, Pooja Gajbhiye, and Vaibhav Rasekar. "Meher (Medical Help Robo)." Annals of the Romanian Society for Cell Biology (2020): 298-307.}
see \cite{rahangdale2020meher}

The main objective of this project is tofabricate a robotic trolley for material handling inindustries. In this project a robotic vehicle is fabricated which runs like a car by carrying necessary tools from one place toanother. The motor is connected with the wheel.When the trolley is loaded with a tool or some other goodsit can be easily move to the place as per need by means ofwireless remote controller .It can be used in industries, hospitals etc. This paper describes the evolving role of robotics in healthcare and allied areas with special concerns relating to the management and control of the spread of the viral or contagious diseases.  The prime utilization of such robots is to minimize person-to-person contact and to ensure cleaning, sterilization and support in hospitals and similar facilities such as quarantine. This will result in minimizing the life threat to medical staff and doctors taking an active role in the management of such diseases. The intention of the present research is to highlight the importance of medical robotics in general and then to connect its utilization with the perspective of viral disease management so that the hospital management can direct themselves to maximize the use of medical robots for various medical procedures. This is despite the popularity of telemedicine, which is also effective in similar situations.

\medskip
\subsubsection{Ruby, J., Susan Daenke, Xianpei Li, J. Tisa, William Harry, J. Nedumaan, Mingmin Pan, J. Lepika, Thomas Binford, and PS Jagadeesh Kumar. "Integrating medical robots for brain surgical applications." J Med Surg 5, no. 1 (2020): 1-14.}
see \cite{ruby2020integrating}

Neurosurgery has customarily been at the cutting edge of propelling innovations, adjusting new strategies and gadgets effectively with an end goal to build the security and viability of brain surgery procedures. Among these adjustments is the surgical robot technology. This paper features a portion of the all the more encouraging frameworks in neurosurgical robotics, integrating brain surgical robots being used and being advanced. The reason for this paper is twofold, to address the most encouraging models for neurosurgical applications, and to examine a portion of the entanglements of robotic neurosurgery given the exceptional framework of the brain. The utilization of robotic assistance and input direction on surgical operations could improve the specialization of the experts during the underlying period of the expectation to absorb information.

\medskip
\subsubsection{Petersen, Inga Lypunova, Weronika Nowakowska, Christian Ulrich, and Lotte NS Andreasen Struijk. "A Novel sEMG Triggered FES-Hybrid Robotic Lower Limb Rehabilitation System for Stroke Patients." IEEE Transactions on Medical Robotics and Bionics 2, no. 4 (2020): 631-638.} see \cite{petersen2020novel}

Stroke is a leading cause of acquired disability among adults. Current rehabilitation programs result in only partial recovery of motor ability for the patients, which has resulted in an ongoing search for methods to improve the rehabilitation approaches. Therefore, this study presents a novel method for early onset of active rehabilitation by combining an end effector robot with surface electromyography (sEMG) triggered functional electrical stimulation (FES) of rectus femoris and tibialis anterior muscles. This rehabilitation system was demonstrated in 10 able-bodied experimental participants. Defining a successful exercise repetition as a fully completed exercise, from start point to end point followed by a return to start point, when FES onset is triggered by the EMG threshold, the results showed that 97 \% of the exercise repetitions were successful for a leg press exercise and 100 \% for a dorsiflexion exercise. Furthermore, an FES stimulation current amplitude of 20–53 mA was required for the leg press exercise and 10–30 mA for the dorsiflexion exercise. 

\medskip
\subsubsection{Atallah, Asa B., and Sam Atallah. "Cloud Computing for Robotics and Surgery." In Digital Surgery, pp. 37-58. Springer, Cham, 2021.}
see \cite{atallah2021cloud}

This chapter is intended to be an introduction to cloud computing for surgeons and noncomputer scientists. In addition to presenting a modern history of the cloud, it explores theoretical concepts of applying cloud computer systems to next-generation medical robots and operating room infrastructures. It explains how the cloud is suited for high-scale computational tasks necessary for the integration of artificial intelligence and machine learning into tomorrow’s surgical suite and how it will provide a framework for digital surgery. Machine learning via the cloud versus single machine learning is also addressed.

\medskip
\subsubsection{Mathis-Ullrich, F., and P. M. Scheikl. "Robots in the operating room-(co) operation during surgery." Der Gastroenterologe: Zeitschrift fur Gastroenterologie und Hepatologie (2020): 1-8.}
see \cite{mathis2020robots}

Background

Medical robotics has the potential to improve surgical and endoluminal procedures by enabling high-precision movements and superhuman perception.

Objectives

To present historical, existing and future robotic assistants for surgery and to highlight their characteristics and advantages for keyhole surgery and endoscopy.

Methods

In particular, historical medical robots and conventional telemanipulators are presented and compared with minimally invasive continuum robots and novel robotic concepts from practice and research. In addition, a perspective for future generations of surgical and endoluminal robots is offered.

Conclusion

Robot-assisted medicine offers great added value for quality of intervention as well as safety for surgeons and patients. In the future, more surgical steps will be performed (semi-)autonomously and in cooperation with the experts.

\medskip
\subsubsection{Dey, Sharmita, Takashi Yoshida, Robert H. Foerster, Michael Ernst, Thomas Schmalz, and Arndt F. Schilling. "Continuous Prediction of Joint Angular Positions and Moments: A Potential Control Strategy for Active Knee-Ankle Prostheses." IEEE Transactions on Medical Robotics and Bionics 2, no. 3 (2020): 347-355.}
see \cite{dey2020continuous}

Transfemoral amputation substantially impairs locomotion. To restore the lost locomotive capability, amputees rely on knee-ankle prostheses. Theoretically, active knee-ankle prostheses may better support natural gait than their passive counterparts by replacing the missing muscle function. The control algorithms of such active devices need to comprehend the user's locomotive intention and convert them into control commands for actuating the prosthesis. For an active knee-ankle prosthesis, the gait variables to be controlled to allow the desired locomotion could be the knee angle, knee moment, ankle angle, ankle moment. In this paper, a random forest regression model is employed for the continuous prediction of these gait variables for level ground walking at self-selected normal speed. Experimentally obtained thigh kinematics were the input to the random forest model. The proposed method could predict the angles and moments of the knee and ankle with high accuracy (mean R2 value of 0.97 for ankle angle, 0.98 for ankle moment, 0.99 for knee angle, 0.95 for knee moment across four able-bodied subjects). The proposed strategy shows potential for continuously controlling an active knee-ankle prosthesis for transfemoral amputees, whose thigh angular motion can be used to infer the required prosthetic moments or angles

\medskip
\subsubsection{Leporini, Alice, Elettra Oleari, Carmela Landolfo, Alberto Sanna, Alessandro Larcher, Giorgio Gandaglia, Nicola Fossati et al. "Technical and functional validation of a teleoperated multirobots platform for minimally invasive surgery." IEEE Transactions on Medical Robotics and Bionics 2, no. 2 (2020): 148-156.}
see \cite{leporini2020technical}

Nowadays Robotic assisted Minimally Invasive Surgeries (R-MIS) are the elective procedures for treating highly accurate and scarcely invasive pathologies, thanks to their ability to empower surgeons' dexterity and skills. The research on new Multi-Robots Surgery (MRS) platform is cardinal to the development of a new SARAS surgical robotic platform, which aims at carrying out autonomously the assistants tasks during R-MIS procedures. In this work, we will present the SARAS MRS platform validation protocol, framed in order to assess: (i) its technical performances in purely dexterity exercises, and (ii) its functional performances. The results obtained show a prototype able to put the users in the condition of accomplishing the tasks requested (both dexterity- and surgical-related), even with reasonably lower performances respect to the industrial standard. The main aspects on which further improvements are needed result to be the stability of the end effectors, the depth perception and the vision systems, to be enriched with dedicated virtual fixtures. The SARAS' aim is to reduce the main surgeon's workload through the automation of assistive tasks which would benefit both surgeons and patients by facilitating the surgery and reducing the operation time.

\medskip
\subsubsection{Perez-Guagnelli, Eduardo, Joanna Jones, Ahmet H. Tokel, Nicolas Herzig, Bryn Jones, Shuhei Miyashita, and Dana D. Damian. "Characterization, simulation and control of a soft helical pneumatic implantable robot for tissue regeneration." IEEE Transactions on Medical Robotics and Bionics 2, no. 1 (2020): 94-103.}
see \cite{perez2020characterization}

Therapies for tissue repair and regeneration have remained sub-optimal, with limited approaches investigated to improve their effectiveness, dynamic and control response. We introduce a Soft Pneumatic Helically-Interlayered Actuator (SoPHIA) for tissue repair and regeneration of tubular tissues. The actuator features shape configurability in two and three dimensions for minimal or non-invasive in vivo implantation; multi-modal therapy to apply mechanical stimulation axially and radially, in accordance to the anatomy of tubular tissues; and anti-buckling structural strength. We present a model and characteristics of this soft actuator. SoPHIA reaches up to 36.3 \% of elongation with respect to its initial height and up to 7 N of force when pressurized at 38 kPa against anatomically-realistic spatial constraints. Furthermore, we introduce the capabilities of a physical in vivo simulator of biological tissue stiffness and growth, for the evaluation of the soft actuator in physiologically-relevant conditions. Lastly, we propose a model-based multi-stage control of the axial elongation of the actuator according to the tissue's physiological response. SoPHIA has the potential to reduce the invasiveness of surgical interventions and increase the effectiveness in growing tissue due to its mechanically compliant, configurable and multi-modal design.

\medskip
\subsubsection{Trefry, Elizabeth, and Tennille Gifford. "The Implications of Robots in Health Care."}
see \cite{trefryimplications}

This paper will discuss the historical differences in education for healthcare staff and the evolution of training and preparing the healthcare worker using technology in today’s healthcare environment. The trainee no longer has to rely on their ability to pretend certain circumstances are occurring, the scenario can be created using technology, artificial intelligence and virtual reality. The conclusion of the paper will review the benefits and risks of relying on technology versus skills learned “in the real world”.

\medskip
\subsubsection{Bayro-Corrochano, Eduardo. "Geometric Computing for Minimal Invasive Surgery." In Geometric Algebra Applications Vol. II, pp. 565-583. Springer, Cham, 2020.}
see \cite{bayro2020geometric}

In this chapter, we show the treatment of a variety of tasks of medical robotics handled using a powerful, non-redundant coefficient geometric language. This chapter is based on our previous works [1, 2]. You will see how we can treat the representation and modeling using geometric primitives like points, lines, and spheres. The screw and motors are used for interpolation, grasping, holding, object manipulation, and surgical maneuvering. We use geometric algebra algorithms in three scenarios: the virtual world for surgical planning, the haptic interface to command the robot arms, and the visually guided robot arms system for operation of ultrasound scanning and surgery. Note that in this work, we do not present a complete system for computer-aided surgery, here we illustrate the application of geometric algebra algorithms for some relevant tasks in minimal invasive surgery.

\medskip
\subsubsection{Sharon, Yarden, Anthony M. Jarc, Thomas S. Lendvay, and Ilana Nisky. "Rate of Orientation Change as a New Metric for Robot-Assisted and Open Surgical Skill Evaluation." IEEE Transactions on Medical Robotics and Bionics 3, no. 2 (2021): 414-425.}
see \cite{sharon2021rate}

Surgeons’ technical skill directly impacts patient outcomes. To date, the angular motion of the instruments has been largely overlooked in objective skill evaluation. To fill this gap, we have developed metrics for surgical skill evaluation that are based on the orientation of surgical instruments. We tested our new metrics on two datasets with different conditions: (1) a dataset of experienced robotic surgeons and nonmedical users performing needle-driving on a dry lab model, and (2) a small dataset of suturing movements performed by surgeons training on a porcine model. We evaluated the performance of our new metrics (angular displacement and the rate of orientation change) alongside the performances of classical metrics (task time and path length). We calculated each metric on different segments of the movement. Our results highlighted the importance of segmentation rather than calculating the metrics on the entire movement. Our new metric, the rate of orientation change, showed statistically significant differences between experienced surgeons and nonmedical users / novice surgeons, which were consistent with the classical task time metric. The rate of orientation change captures technical aspects that are taught during surgeons’ training, and together with classical metrics can lead to a more comprehensive discrimination of skills.


\medskip
\subsubsection{Boehler, Quentin, David S. Gage, Phyllis Hofmann, Alexandra Gehring, Christophe Chautems, Donat R. Spahn, Peter Biro, and Bradley J. Nelson. "REALITI: A robotic endoscope automated via laryngeal imaging for tracheal intubation." IEEE Transactions on Medical Robotics and Bionics 2, no. 2 (2020): 157-164.}
see \cite{boehler2020realiti}

Tracheal intubation is considered the gold standard to secure the airway of patients in need of respiratory assistance, yet this procedure relies on the dexterity and experience of the physician to correctly place a tracheal tube into the patient's trachea. Such a complex procedure may greatly benefit from robotic assistance in order to make the intubation safer and more efficient. We developed the first device to provide such assistance, the REALITI, which stands for Robotic Endoscope Automated via Laryngeal Imaging for Tracheal Intubation. This device allies the automated detection of key anatomical features in an endoscopic image to the robotic steering toward the recognized features in the task of guiding the tracheal tube into its correct position. The pre-clinical prototype presented in this paper has been developed to perform in vitro tracheal inbutation on a standard airway management training manikin. We performed a robust detection of anatomical features to steer the endoscope in a visual servoing fashion. Our prototype has been successfully used to perform automated and manual insertions into the trachea of an airway manikin.

\medskip
\subsubsection{Langlois, Kevin, David Rodriguez-Cianca, Ben Serrien, Joris De Winter, Tom Verstraten, Carlos Rodriguez-Guerrero, Bram Vanderborght, and Dirk Lefeber. "Investigating the Effects of Strapping Pressure on Human-Robot Interface Dynamics Using a Soft Robotic Cuff." IEEE Transactions on Medical Robotics and Bionics 3, no. 1 (2020): 146-155.}
see \cite{langlois2020investigating}

Physical human-robot interfaces are a challenging aspect of exoskeleton design, mainly due to the fact that interfaces tend to migrate relatively to the body leading to discomfort and power losses. Therefore, the key is to develop interfaces that optimize attachment stiffness, i.e., reduce relative motion, without compromising comfort. To that end, we propose a method to obtain the optimal attachment pressure in terms of connection stiffness and comfort. The method is based on a soft robotic interface capable of actively controlling strapping pressure which is coupled to a cobot. Hereby the effects of strapping pressure on energetic losses, connection stiffness, and perceived comfort are analyzed. Results indicate a trade-off between connection stiffness and perceived comfort for this type of interface. An optimal strapping pressure was found in the 50 to 80 mmHg range. Connection stiffness was found to increase linearly over a pressure range from 0 mmHg (stiffness of 1139 N/m) to 100 mmHg (stiffness of 2232 N/m). And energetic losses were reduced by 42\% by increasing connection stiffness. This research highlights the importance of strapping pressure when attaching an exoskeleton to a human and introduces a new adaptive interface to improve the coupling from an exoskeleton to an individual

\medskip
\subsubsection{Wang, Chao, Hao Zhang, Lu Zhang, Meng Kong, Kai Zhu, Chuan‐li Zhou, and Xue‐xiao Ma. "Accuracy and deviation analysis of robot‐assisted spinal implants: A retrospective overview of 105 cases and preliminary comparison to open freehand surgery in lumbar spondylolisthesis." The International Journal of Medical Robotics and Computer Assisted Surgery (2021): e2273.}
see \cite{wang2021accuracy}

Background

Whether the accuracy of robot-assisted spinal screw placement is significantly higher than that of freehand and the source of robotic deviation remain unclear.

Methods

Clinical data of 105 patients who underwent robot-assisted spinal surgery was collected, and screw accuracy was evaluated by computed tomography according to the modified Gertzbein–Robbins classification. Patients were grouped by percutaneous and open surgery. Intergroup comparisons of clinical and screw accuracy parameters were performed. Reasons for deviation were determined. Thirty-one patients with lumbar spondylolisthesis undergoing open robot-assisted surgery and the same number of patients treated by open freehand surgery were compared for screw accuracy.

Results

Screw accuracy was not significantly different between the percutaneous and open groups in both intra- and postoperative evaluations. Tool skiving was identified as the main cause of deviation. The proportion of malpositioned screws (grade B + C + D) was significantly higher in the freehand group than in the robot-assisted group. However, remarkably malpositioned (grade C + D) screws showed no significant differences between the groups. No revision surgery was necessary.

Conclusions

Robot-assisted spinal instrumentation manifests high accuracy and low incidence of nerve injury. Tool skiving is a major cause of implant deviation.

\medskip
\subsubsection{Kastritsi, Theodora, and Zoe Doulgeri. "A Controller to Impose a RCM for Hands-on Robotic-Assisted Minimally Invasive Surgery." IEEE Transactions on Medical Robotics and Bionics 3, no. 2 (2021): 392-401.}
see \cite{kastritsi2021controller}

In Robotic-Assisted Minimally Invasive Surgery a long and thin instrument attached to the robotic arm enters the human body through a tiny incision. To ensure that no injury occurs when the surgeon is manipulating the instrument, the incision point must be a remote center of motion (RCM) for the instrument. For this purpose, a novel target admittance model is designed in the joint space for hands-on procedures that can be applied in all commercially available general-purpose manipulators with six or more degrees of freedom. It is proved that the joint reference trajectories generated by the proposed target admittance model under the exertion of a human force are stable and satisfy the RCM constraint. The measurements of the human force and the robot’s forward kinematic model are only required. Its use spans all hands-on surgical procedures. The proposed model can be easily extended to achieve additional objectives. Simulation results validate the theoretical findings and experimental results utilizing a KUKA LWR4+ demonstrate that trocar displacements are less than 1mm.

\medskip
\subsubsection{Pradhan, Bikash, Deepti Bharti, Sumit Chakravarty, Sirsendu S. Ray, Vera V. Voinova, Anton P. Bonartsev, and Kunal Pal. "Internet of Things and Robotics in Transforming Current-Day Healthcare Services." Journal of Healthcare Engineering 2021 (2021).}
see \cite{pradhan2021internet}

Technology has become an integral part of everyday lives. Recent years have witnessed advancement in technology with a wide range of applications in healthcare. However, the use of the Internet of Things (IoT) and robotics are yet to see substantial growth in terms of its acceptability in healthcare applications. The current study has discussed the role of the aforesaid technology in transforming healthcare services. The study also presented various functionalities of the ideal IoT-aided robotic systems and their importance in healthcare applications. Furthermore, the study focused on the application of the IoT and robotics in providing healthcare services such as rehabilitation, assistive surgery, elderly care, and prosthetics. Recent developments, current status, limitations, and challenges in the aforesaid area have been presented in detail. The study also discusses the role and applications of the aforementioned technology in managing the current pandemic of COVID-19. A comprehensive knowledge has been provided on the prospect of the functionality, application, challenges, and future scope of the IoT-aided robotic system in healthcare services. This will help the future researcher to make an inclusive idea on the use of the said technology in improving the healthcare services in the future.

\medskip
\subsubsection{Xu, Yongkang, Hongqiang Zhao, Xuanyi Zhou, Salih Ertug Ovur, and Ting Xia. "Energy Management for Medical Rescue Robot." In 2020 5th International Conference on Advanced Robotics and Mechatronics (ICARM), pp. 44-49. IEEE, 2020.}
see \cite{xu2020energy}

This paper mainly centers on the energy management of the medical rescue mobile robot with different payloads in the uncertain road conditions. Efficient energy usage is a crucial issue because energy consumption will increase with the application and expansion of robotics. The power consumption of the robot system is affected by the task and its operating environment. This paper designs a six-wheel-drive medical rescue mobile robot using a hybrid system. Aiming at the wheel driving conditions of mobile robots, a power-adaptation control strategy is proposed, and its structure and driving form is introduced. Firstly, based on a power-adaptation control strategy, a nonlinear model predictive control (NMPC) algorithm is proposed to optimize the power adaptive control strategy. Non-linear model predictive control uses a non-linear model to represent the medical rescue robot model and external characteristic constraints. Secondly, the AVL CRUISE module is used to build a dynamic model of a medical rescue mobile robot, and it is jointly simulated with MATLAB/Simulink to simulate the energy consumption of a medical rescue robot under Urban Driving Cycle (UDC) standard operating conditions. Finally, the NMPC method is used to solve the system and compared with the power-adaptation control strategy. The simulation results show that compared with the power adaptive control strategy, the fuel consumption of combustion engine of the NMPC is improved by 26.4\%.

\medskip
\subsubsection{Su, Baiquan, Shi Yu, Xintong Li, Yi Gong, Han Li, Zifeng Ren, Yijing Xia et al. "Autonomous Robot for Removing Superficial Traumatic Blood." IEEE Journal of Translational Engineering in Health and Medicine 9 (2021): 1-9.}
see \cite{su2021autonomous}

Objective: To remove blood from an incision and find the incision spot is a key task during surgery, or else over discharge of blood will endanger a patient's life. However, the repetitive manual blood removal involves plenty of workload contributing fatigue of surgeons. Thus, it is valuable to design a robotic system which can automatically remove blood on the incision surface. Methods: In this paper, we design a robotic system to fulfill the surgical task of the blood removal. The system consists of a pair of dual cameras, a 6-DoF robotic arm, an aspirator whose handle is fixed to a robotic arm, and a pump connected to the aspirator. Further, a path-planning algorithm is designed to generate a path, which the aspirator tip should follow to remove blood. Results: In a group of simulating bleeding experiments on ex vivo porcine tissue, the contour of the blood region is detected, and the reconstructed spatial coordinates of the detected blood contour is obtained afterward. The BRR robot cleans thoroughly the blood running out the incision. Conclusions: This study contributes the first result on designing an autonomous blood removal medical robot. The skill of the surgical blood removal operation, which is manually operated by surgeons nowadays, is alternatively grasped by the proposed BRR medical robot.

\medskip
\subsubsection{Avila-Tomás, J. F., M. A. Mayer-Pujadas, and V. J. Quesada-Varela. "Artificial intelligence and its applications in medicine I: introductory background to AI and robotics." Atencion Primaria 52, no. 10 (2020): 778-784.}
see \cite{avila2020artificial}

Technology and medicine follow a parallel path during the last decades. Technological advances are changing the concept of health and health needs are influencing the development of technology. Artificial intelligence (AI) is made up of a series of sufficiently trained logical algorithms from which machines are capable of making decisions for specific cases based on general rules. This technology has applications in the diagnosis and follow-up of patients with an individualized prognostic evaluation of them. Furthermore, if we combine this technology with robotics, we can create intelligent machines that make more efficient diagnostic proposals in their work. Therefore, AI is going to be a technology present in our daily work through machines or computer programs, which in a more or less transparent way for the user, will become a daily reality in health processes. Health professionals have to know this technology, its advantages and disadvantages, because it will be an integral part of our work. In these two articles we intend to give a basic vision of this technology adapted to doctors with a review of its history and evolution, its real applications at the present time and a vision of a future in which AI and Big Data will shape the personalized medicine that will characterize the 21st century.

\medskip
\subsubsection{Maglio, S., C. Park, S. Tognarelli, A. Menciassi, and E. T. Roche. "High fidelity physical organ simulators: from artificial to bio hybrid solutions." IEEE Transactions on Medical Robotics and Bionics (2021).}
see \cite{maglio2021high}

Over the past decade, there has been growing interest in high-fidelity simulation for medical applications leading to huge research efforts towards physical organ simulators with realistic representations of human organs. As this is a relatively young research field, this review aims to provide an insight into the current state of the art in high-fidelity physical organ simulators that are used for training purposes, as educational tools, for biomechanical studies, and for preclinical device testing. Motivated by a paucity of clear definitions and categorization of various simulators, we describe high-fidelity physical organ simulators in terms of their degree of representation of the anatomy, material properties, and physiological behavior of the target organs in the context of their applications. We highlight the traditional approaches for static organ simulators using synthetic materials, and diverse approaches for dynamic organ simulators including soft robotic, ex vivo , and biohybrid strategies to meet the ever-increasing demand for realistic anthropomorphic organ models. Finally, we discuss challenges and potential future avenues in the field of high-fidelity physical organ simulators.

\medskip
\subsubsection{Xia, Runzhi, Zhicheng Tong, Yi Hu, Keyu Kong, Xiulin Wu, and Huiwu Li. "“Skywalker” surgical robot for total knee arthroplasty: An experimental sawbone study." The International Journal of Medical Robotics and Computer Assisted Surgery (2021): e2292.}
see \cite{xia2021skywalker}

Background

Currently, robot-assisted surgical systems are used to reduce the error range of total knee arthroplasty (TKA) osteotomy and component positioning.

Methods

We used 20 sawbone models of the femur and 20 sawbone models of the tibia and fibula to evaluate the osteotomy effect of ‘Skywalker’ robot-assisted TKA.

Results

The maximal movement of the cutting jig was less than 0.25 mm at each osteotomy plane. The mean and standard deviation values of the angle deviation between the planned osteotomy plane and the actual osteotomy plane at each osteotomy plane were not more than 1.03° and 0.55°, respectively. The mean and standard deviation values of absolute error of resection thickness at each osteotomy position were less than 0.78 and 0.71 mm, respectively.

Conclusions

The ‘Skywalker’ system has good osteotomy accuracy, can achieve the planned osteotomy well and is expected to assist surgeons in performing accurate TKA in clinical applications in future.

\medskip
\subsubsection{Lan, Ning, Manzhao Hao, Chuanxin M. Niu, He Cui, Yu Wang, Ting Zhang, Peng Fang, and Chih-hong Chou. "Next-generation prosthetic hand: from biomimetic to biorealistic." Research 2021 (2021).}
see \cite{lan2021next}

Integrating a prosthetic hand to amputees with seamless neural compatibility presents a grand challenge to neuroscientists and neural engineers for more than half century. Mimicking anatomical structure or appearance of human hand does not lead to improved neural connectivity to the sensorimotor system of amputees. The functions of modern prosthetic hands do not match the dexterity of human hand due primarily to lack of sensory awareness and compliant actuation. Lately, progress in restoring sensory feedback has marked a significant step forward in improving neural continuity of sensory information from prosthetic hands to amputees. However, little effort has been made to replicate the compliant property of biological muscle when actuating prosthetic hands. Furthermore, a full-fledged biorealistic approach to designing prosthetic hands has not been contemplated in neuroprosthetic research. In this perspective article, we advance a novel view that a prosthetic hand can be integrated harmoniously with amputees only if neural compatibility to the sensorimotor system is achieved. Our ongoing research supports that the next-generation prosthetic hand must incorporate biologically realistic actuation, sensing, and reflex functions in order to fully attain neural compatibility.

\medskip
\subsubsection{Kunz, Christian, Michal Hlaváč, Max Schneider, Andrej Pala, Pit Henrich, Birgit Jickeli, Heinz Wörn, Björn Hein, Rainer Wirtz, and Franziska Mathis-Ullrich. "Autonomous Planning and Intraoperative Augmented Reality Navigation for Neurosurgery." IEEE Transactions on Medical Robotics and Bionics (2021).}
see \cite{kunz2021autonomous}

Neurosurgical interventions in the brain are challenging due to delicate anatomical structures. During surgery, precise navigation of surgical instruments supports surgeons and allows prevention of adverse events. Here, an augmented reality-based navigation aid with automated segmentation of risk structures and path planning is presented. Superimposed patient models are visualized during neurosurgical interventions on the example of the ventricular puncture. The proposed system is experimentally validated in a realistic operating room scenario with expert neurosurgeons to determine its quality of support as well as its potential for clinical translation. The automated segmentation reaches a F1-Score of 95-99\%. Paths are planned correctly in 93.4\%. The entire process enables navigation aid in under five minutes. Validation shows that the system allows for a puncture success rate of 81.7\% with mean accuracy of 4.8 ± 2.5 mm. A control group who performed the standard-of-care procedure reached a rate of 71.7\% with 6.5 ± 2.4 mm accuracy. Acceptability analysis shows that 85.7\% of the participating surgeons approve of the system’s convenience and 92.9\% expect accuracy improvement. The presented navigation aid for ventricular puncture enables automated surgical planning and may improve accuracy and success rates of neurosurgical interventions.

\medskip
\subsubsection{Satale, Kavita, Tanmayi Bhave, Chirag Chandak, and S. A. Patil. "Nursing Robot." (2020).}
see \cite{satale2020nursing}

This paper highlights the role of robotics in healthcare .The paper also focuses on the areas of management in the hospital and control of the spread of the novel coronavirus disease 2019 (COVID-19). The main intension of such robots is to minimize person-to-person contact and also to ensure cleaning, sterilization and support in hospitals and similar facilities such as quarantine. This will be useful for reducing the life threat to doctors and medical staff taking an active role in the management of the COVID-19 pandemic. The purpose of the present research on robot is to highlight the importance of medical robotics in general, and then to connect its utilization with the intension of COVID-19 management so that the hospital staff can direct themselves to maximize the use of medical robots for various medical procedures. This is despite the popularity of telemedicine robots, which are also effective in similar situations? Our proposed system will help nurses and doctors to supply medicines as well as food to infected patients

\medskip
\subsubsection{Zhang, Jing, Jiahui Qian, Han Zhang, Ling He, Bin Li, Jing Qin, Hongning Dai, Wei Tang, and Weidong Tian. "Maxillofacial surgical simulation system with haptic feedback." Journal of Industrial \& Management Optimization (2020).}
see \cite{zhang2020maxillofacial}

Due to the complexity of the maxillofacial surgery, the novice should be sufficiently trained before one is qualified to carry on the surgery. To reduce the training costs and improve the training efficiency, a virtual mandible surgical system with haptic feedback is proposed. This surgical simulation system offers users the haptic feedback while simulating maxillofacial surgery. An integrated model is introduced to optimize the system simulation process, which includes force output to a six-degree-of-freedom haptic device. Based on the anatomy structure of the bone tissue, a two-layer mechanism model is designed to balance the requirement of real-time response and the force feedback accuracy. Collision detection, force rendering, and grinding function are studied to simulate some essential operations: open reduction, osteotomy, and palate fixation. The proposed simulation platform can assist in the training and planning of these oral and maxillofacial surgeries. The fast response feature enables surgeons to design a patient-specific guide plate in real-time. Ten stomatology surgeons evaluated this surgical simulation system from the following four indexes: the level of immersion, user-friendliness, stability, and the effect of surgical training. The evaluation score is eight out of ten.


\medskip
\subsubsection{Caccianiga, Guido, Andrea Mariani, Elena De Momi, Gabriela Cantarero, and Jeremy D. Brown. "An evaluation of inanimate and virtual reality training for psychomotor skill development in robot-assisted minimally invasive surgery." IEEE Transactions on Medical Robotics and Bionics 2, no. 2 (2020): 118-129.}
see \cite{caccianiga2020evaluation}

Robot-assisted minimally invasive surgery (RAMIS) is gaining widespread adoption in many surgical specialties, despite the lack of a standardized training curriculum. Current training approaches rely heavily on virtual reality simulators, in particular for basic psychomotor and visuomotor skill development. It is not clear, however, whether training in virtual reality is equivalent to inanimate model training. In this manuscript, we seek to compare virtual reality training to inanimate model training, with regard to skill learning and skill transfer. Using a custom-developed needle-driving training task with inanimate and virtual analogs, we investigated the extent to which N=18 participants improved their skill on a given platform post-training, and transferred that skill to the opposite platform. Results indicate that the two approaches are not equivalent, with more salient skill transfer after inanimate training than virtual training. These findings support the claim that training with real physical models is the gold standard, and suggest more inanimate model training be incorporated into training curricula for early psychomotor skill development.

\medskip
\subsubsection{Su, Hang, Yunus Schmirander, Sarah Elena Valderrama-Hincapie, Jairo Pinedo, Xuanyi Zhou, Jiehao Li, Longbin Zhang, Yingbai Hu, Giancarlo Ferrigno, and Elena De Momi. "Asymmetric bimanual control of dual-arm serial manipulator for robot-assisted minimally invasive surgeries." (2020): 1223-1233.}
see \cite{su2020asymmetric}

Robotic assistance is promising for improving minimally invasive surgery (MIS). This work presents asymmetric bimanual control of a dual-arm serial robot with two remote centers of motion (RCMs) constraints for MIS. In our previous works, general null space controllers to guarantee the fixed RCM constraint have been proposed. However, an incision on a patient’s abdominal wall is not fixed owing to the respiration of the patient, which generates an uncertain disturbance at the joints of robotic manipulators. To improve accuracy, a radial basis function neural network is implemented to adapt to these disturbances and control the end-effector position. Finally, the adaptive bimanual control strategy is validated through simulations based on clinical data. The proposed control shows improved accuracy in the end effector position for all the designed surgical tasks. In future works, the algorithm will be validated on an actual dual-arm serial robot making use of a body phantom.

\medskip
\subsubsection{De Rossi, Giacomo, Marco Minelli, Serena Roin, Fabio Falezza, Alessio Sozzi, Federica Ferraguti, Francesco Setti, Marcello Bonfè, Cristian Secchi, and Riccardo Muradore. "A First Evaluation of a Multi-Modal Learning System to Control Surgical Assistant Robots via Action Segmentation." IEEE Transactions on Medical Robotics and Bionics (2021).}
see \cite{de2021first}

The next stage for robotics development is to introduce autonomy and cooperation with human agents in tasks that require high levels of precision and/or that exert considerable physical strain. To guarantee the highest possible safety standards, the best approach is to devise a deterministic automaton that performs identically for each operation. Clearly, such approach inevitably fails to adapt itself to changing environments or different human companions. In a surgical scenario, the highest variability happens for the timing of different actions performed within the same phases. This paper presents a cognitive control architecture that uses a multi-modal neural network trained on a cooperative task performed by human surgeons and produces an action segmentation that provides the required timing for actions while maintaining full phase execution control via a deterministic Supervisory Controller and full execution safety by a velocity-constrained Model-Predictive Controller.

\medskip
\subsubsection{Thai, Mai Thanh, Phuoc Thien Phan, Trung Thien Hoang, Shing Wong, Nigel H. Lovell, and Thanh Nho Do. "Advanced intelligent systems for surgical robotics." Advanced Intelligent Systems 2, no. 8 (2020): 1900138.}
see \cite{thai2020advanced}

Surgical robots have had clinical use since the mid-1990s. Robot-assisted surgeries offer many benefits over the conventional approach including lower risk of infection and blood loss, shorter recovery, and an overall safer procedure for patients. The past few decades have shown many emerging surgical robotic platforms that can work in complex and confined channels of the internal human organs and improve the cognitive and physical skills of the surgeons during the operation. Advanced technologies for sensing, actuation, and intelligent control have enabled multiple surgical devices to simultaneously operate within the human body at low cost and with more efficiency. Despite advances, current surgical intervention systems are not able to execute autonomous tasks and make cognitive decisions that are analogous to those of humans. Herein, the historical development of surgery from conventional open to robotic-assisted approaches with discussion on the capabilities of advanced intelligent systems and devices that are currently implemented in existing surgical robotic systems is reviewed. Also, available autonomous surgical platforms are comprehensively discussed with comments on the essential technologies, existing challenges, and suggestions for the future development of intelligent robotic-assisted surgical systems toward the achievement of fully autonomous operation.

\medskip
\subsubsection{Watson, Connor, and Tania K. Morimoto. "Permanent magnet-based localization for growing robots in medical applications." IEEE Robotics and Automation Letters 5, no. 2 (2020): 2666-2673.}
see \cite{watson2020permanent}

Growing robots that achieve locomotion by extending from their tip, are inherently compliant and can safely navigate through constrained environments that prove challenging for traditional robots. However, the same compliance and tip-extension mechanism that enables this ability, also leads directly to challenges in their shape estimation and control. In this letter, we present a low-cost, wireless, permanent magnet-based method for localizing the tip of these robots. A permanent magnet is placed at the robot tip, and an array of magneto-inductive sensors is used to measure the change in magnetic field as the robot moves through its workspace. We develop an approach to localization that combines analytical and machine learning techniques and show that it outperforms existing methods. We also measure the position error over a 500mm × 500 mm workspace with different magnet sizes to show that this approach can accommodate growing robots of different scales. Lastly, we show that our localization method is suitable for tracking the tip of a growing robot by deploying a 12 mm robot through different, constrained environments. Our method achieves position and orientation errors of 3.0 ± 1.1 mm and 6.5 ±5.4° in the planar case and 4.3 ± 2.3 mm, 3.9 ±3.0°, and 3.8 ±3.5° in the 5-DOF setting.

\medskip
\subsubsection{Nawrat, Zbigniew. "MIS AI-artificial intelligence application in minimally invasive surgery." Mini-invasive Surgery 4 (2020).}
see \cite{nawrat2020mis}

This chapter is devoted towards analyzing the progress and barriers to the development of artificial intelligence (AI) and medical robotics in minimally-invasive surgery. The less invasive the surgical intervention and the further the surgeon is from the operating table, the greater the roles of decision support systems (AI) and performance of specific tasks (by medical robots).

\medskip
\subsubsection{Li, Mi, Ning Xi, Yuechao Wang, and Lianqing Liu. "Progress in nanorobotics for advancing biomedicine." IEEE Transactions on Biomedical Engineering 68, no. 1 (2020): 130-147.}
see \cite{li2020progress}

Nanorobotics, which has long been a fantasy in the realm of science fiction, is now a reality due to the considerable developments in diverse fields including chemistry, materials, physics, information and nanotechnology in the past decades. Not only different prototypes of nanorobots whose sizes are nanoscale are invented for various biomedical applications, but also robotic nanomanipulators which are able to handle nano-objects obtain substantial achievements for applications in biomedicine. The outstanding achievements in nanorobotics  have significantly expanded the field of medical robotics and yielded novel insights into the underlying mechanisms guiding life activities, remarkably showing an emerging and promising way for advancing the diagnosis \& treatment level in the coming era of personalized precision medicine. In this review, the recent advances in nanorobotics (nanorobots, nanorobotic manipulations) for biomedical applications are summarized from several facets (including molecular machines, nanomotors, DNA nanorobotics, and robotic nanomanipulators), and the future perspectives are also presented.

\medskip
\subsubsection{Giesen, Luuk, Laurie Bax, and Jurgen Riedl. "Automated real-time 3D ultrasound mapping of vessels [3D-ULTRAMAN]."}
see \cite{giesenautomated}

The 3D Ultraman project is a collaboration between medical robotics company Vitestro and clinical laboratory Result Laboratorium.
During the project, real-time detection of arteries were developed, including a clinical ultrasound-force study as well as training and
evaluation of deep learning architectures. In a phase 2 project, these promising algorithms will be further developed and integrated in an
ATTRACT phase 2 project. This technology could be the basis for robotic devices autonomously performing a variety of vascular access
procedures, enabling new levels of automation in medicine.

\medskip
\subsubsection{Cortesão, Rui, and Luís Santos. "Noise Effects on Quaternion and Axis-Angle Representations in Robotics." IEEE Robotics and Automation Letters 6, no. 1 (2020): 64-71.}
see \cite{cortesao2020noise}

This letter provides a methodology to analyze noise sensitivity of quaternion and axis-angle representations in the presence of joint measurement noise. A general formulation based on the trace of the rotation matrix is proposed, enabling to compute noise sensitivity as a function of robot postures and noise variances. Additionally, noise sensitivity as a function of the orientation angle is derived, enabling to identify regions with different sensitivities. The theoretical findings are general and are not associated to any particular noise distribution. Simulation results with zero-mean Gaussian noise and real experiments are provided corroborating the theoretical findings.

\medskip
\subsubsection{Yang, Su, Hyuck‐Jun Yoon, Seyed Jamaleddin Mostafavi Yazdi, and Jong‐Ha Lee. "A novel automated lumen segmentation and classification algorithm for detection of irregular protrusion after stents deployment." The International Journal of Medical Robotics and Computer Assisted Surgery 16, no. 1 (2020): e2033.}
see \cite{yang2020novel}

Background

Clinically, irregular protrusions and blockages after stent deployment can lead to significant adverse outcomes such as thrombotic reocclusion or restenosis. In this study, we propose a novel fully automated method for irregular lumen segmentation and normal/abnormal lumen classification.

Methods

The proposed method consists of a lumen segmentation, feature extraction, and lumen classification. In total, 92 features were extracted to classify normal/abnormal lumen. The lumen classification method is a combination of supervised learning algorithm and feature selection that is a partition-membership filter method.

Results

As the results, our proposed lumen segmentation method obtained the average of dice similarity coefficient (DSC) and the accuracy of proposed features and the random forest (RF) for normal/abnormal lumen classification as 97.6\% and 98.2\%, respectively.

Conclusions

Therefore, we can lead to better understanding of the overall vascular status and help to determine cardiovascular diagnosis.

\medskip
\subsubsection{Hwang, Minho, Daniel Seita, Brijen Thananjeyan, Jeffrey Ichnowski, Samuel Paradis, Danyal Fer, Thomas Low, and Ken Goldberg. "Applying depth-sensing to automated surgical manipulation with a Da Vinci Robot." In 2020 International Symposium on Medical Robotics (ISMR), pp. 22-29. IEEE, 2020.}
see \cite{hwang2020applying}

Recent advances in depth-sensing have significantly increased accuracy, resolution, and frame rate, as shown in the 1920x1200 resolution and 13 frames per second Zivid RGBD camera. In this study, we explore the potential of depth sensing for efficient and reliable automation of surgical subtasks. We consider a monochrome (all red) version of the peg transfer task from the Fundamentals of Laparoscopic Surgery training suite implemented with the da Vinci Research Kit (dVRK). We use calibration techniques that allow the imprecise, cable-driven da Vinci to reduce error from 4-5mm to 1-2mm in the task space. We report experimental results for a handover-free version of the peg transfer task, performing 20 and 5 physical episodes with single- and bilateral-arm setups, respectively. Results over 236 and 49 total block transfer attempts for the single- and bilateral-arm peg transfer cases suggest that reliability can be attained with 86.9\% and 78.0\% for each individual block, with respective block transfer speeds of 10.02 and 5.72 seconds. Supplementary material is available at https://sites.google.com/view/peg-transfer.

\medskip
\subsubsection{Sheft, Maxina, Priya Kulkarni, Jiawei Ge, Hamed Saeidi, Justin D. Opfermann, Arjun Joshi, Martin Schnermann, and Axel Krieger. "Development and Error Analysis of a Novel Robotic System for Photodynamic Therapy." In 2020 International Symposium on Medical Robotics (ISMR), pp. 166-172. IEEE, 2020.}
see \cite{sheft2020development}

Photodynamic therapy has the potential to not only treat tumors directly but also to reduce incidental damage caused by large surgical margins and radiation therapy. In this study, a novel robotic system of delivering light was developed using a cartesian robot. Human input was limited to a computer input and no physical positioning of the light delivery system was required during testing. Error analysis was conducted to ensure the system's applicability to a clinical environment. Error involved in both the outlining and coverage of the targeted areas was examined. The average outlining error and standard deviation were 0.23 +/- 0.16mm, and the coverage time error was below 4\%. These results indicate that a robotic light delivery system for photodynamic therapy can consistently provide light delivery with sub-millimeter errors when testing with ex-vivo phantoms.


\medskip
\subsubsection{Li, Junyu, Yanming Fang, Zhao Jin, Yuchen Wang, and Miao Yu. "The impact of robot‐assisted spine surgeries on clinical outcomes: A systemic review and meta‐analysis." The International Journal of Medical Robotics and Computer Assisted Surgery 16, no. 6 (2020): 1-14.}
see \cite{li2020impact}

Background

Medical robotics has enabled a significant advancement in the field of modern spine surgery, especially in pedicle screw fixation. A plethora of studies focused on the accuracy of pedicle fixation in robotic-assisted (RA) technology. However, it is not clear whether RA techniques can improve patients' clinical outcomes.

Methods

We retrieved relevant studies that compare the differences between RA and freehand (FH) techniques in spine surgeries from the following databases: PubMed, Embase, Cochrane Library and Web of Science. The perioperative outcomes of this technology were measured with parameters including radiation exposure, operative time, the length of hospital stay, complication rates and revision rates. Two reviewers independently reviewed the studies in our sample, assessed their validity and extracted relevant data.

Conclusions

This study suggests that RA spine surgeries would result in fewer complications, a lower revision rate and shorter length of hospital stay. As the technology continues to evolve, we may expect more applications of robotic systems in spine surgeries.

\medskip
\subsubsection{Avinash, Apeksha, Alaa Eldin Abdelaal, and Septimiu E. Salcudean. "Evaluation of Increasing Camera Baseline on Depth Perception in Surgical Robotics." In 2020 IEEE International Conference on Robotics and Automation (ICRA), pp. 5509-5515. IEEE, 2020.}
see \cite{avinash2020evaluation}

In this paper, we evaluate the effect of increasing camera baselines on depth perception in robot-assisted surgery. Restricted by the diameter of the surgical trocar through which they are inserted, current clinical stereo endoscopes have a fixed baseline of 5.5 mm. To overcome this restriction, we propose using a stereoscopic "pickup" camera with a side-firing design that allows for larger baselines. We conducted a user study with baselines of 10 mm, 15 mm, 20 mm, and 30 mm to evaluate the effect of increasing baseline on depth perception when used with the da Vinci surgical system. Subjects (N=28) were recruited and asked to rank differently sized poles, mounted at a distance of 200 mm from the cameras, according to their increasing order of height when viewed under different baseline conditions. The results showed that subjects performed better as the baseline was increased with the best performance at a 20 mm baseline. This preliminary proof-of-concept study shows that there is opportunity to improve depth perception in robot-assisted surgical systems with a change in endoscope design philosophy. In this paper, we present this change with our side-firing "pickup" camera and its flexible baseline design. Ultimately, this serves as the first step towards an adaptive baseline camera design that maximizes depth perception in surgery.


\medskip
\subsubsection{Rosero, Hermes Fabian Vargas. "Robotics in surgeryand neurosurgery, applications and challenges, a review." Scientia et Technica 25, no. 3 (2020): 478-490.}
see \cite{rosero2020robotics}

The integration of robots in operating rooms aims to improve the performance and efficiency of various procedures, since it offers remarkable advantages over conventional procedures, in particular precision, hand shake filtering and the possibility of executing complex tasks, however, Considerable challenges still prevail affecting massification and maneuverability on the part of surgeons. In the present work a review of the current state of robotic surgery, the challenges and trends is carried out. Specifically, the need for optimal force feedback mechanisms is evidenced, as well as dynamic visualization through augmented reality or virtual reality. It is not yet possible to determine that robotic surgery has reached standards, however, the integration of alternative technologies will allow surgeons to improve not only the efficiency of the robot, but also of its operation by the surgeon

\medskip
\subsubsection{Sun, Y., J. A. Kim, M. Keshavarz, and A. Thompson. "Microrobots for Precision Medicine."}
see \cite{sunmicrorobots}

The project has visualised and simulated the microrobots behaviour in
response to an external magnetic field, and a 3‐D printed phantom is
going to be used as a replica of the targeted organs. Preliminary results
have also been demonstrated in this poster.


\medskip
\subsubsection{Avgousti, Sotiris, Eftychios G. Christoforou, Andreas S. Panayides, Panicos Masouras, Pierre Vieyres, and Constantinos S. Pattichis. "Robotic systems in current clinical practice." In 2020 IEEE 20th Mediterranean Electrotechnical Conference (MELECON), pp. 269-274. IEEE, 2020.}
see \cite{avgousti2020robotic}

Medical robotic systems are successfully employed in various surgical specialties today. Yet, a substantial number of remarkable systems that have been developed and piloted, have failed to reach commercialization and thus adoption in clinical practice. This is partly due to the strict regulatory requirements, which typically occupy a significant amount of the development time while incurring additional costs. Pertinent to regulatory approvals is the field of Human Factors, which plays a central role in the design of safe and efficient medical devices. This study briefly introduces the FDA regulatory approval process, discusses the role of human factors in the design process and highlights specific robotic systems that have obtained approval for clinical use. The purpose is to show the status of robotic technologies in relation to the current clinical practice.

\medskip
\subsubsection{Zhang, Jian, Weishi Li, Lei Hu, Yu Zhao, and Tianmiao Wang. "A robotic system for spine surgery positioning and pedicle screw placement." The International Journal of Medical Robotics and Computer Assisted Surgery (2021): e2262.}
see \cite{zhang2021robotic}

Background

In recent years, surgeons have explored minimally invasive methods of percutaneous pedicle screw implantation which can effectively reduce human injuries. This article presents an accurate and efficient positioning method and robot system for percutaneous needle placement under c-arm fluoroscopy.

Methods

A simple five degree of freedom (DOF) robot with a unique end-effector is designed to perform perspective calibration and image space registration. The principle of pedicle standard axis positioning is adopted to make the axis of the pedicle overlap with the x-ray axis of c-arm.

Results

Then the clinical operation is carried out to verify the clinical feasibility of the designed robot and positioning method. The experimental results show that a total of 26 pedicle screws were accurately implanted. The accuracy of Grade A is 96.15\%. The positioning time of a single guide pin is about 154.77 s, and three x-ray films need to be taken on average.

Conclusions

The positioning accuracy is increased by using the present method. In addition, this method is simple in operation, short in operation time, low in X-ray exposure.

\medskip
\subsubsection{Yang, Bo, Jian Huang, Xinxing Chen, Caihua Xiong, and Yasuhisa Hasegawa. "Supernumerary Robotic Limbs: A Review and Future Outlook." IEEE Transactions on Medical Robotics and Bionics (2021).}
see \cite{yang2021supernumerary}

Wearable robots have become a prevalent method in the field of human augmentation and medical rehabilitation. Typical wearable robots mainly include exoskeletons and prostheses. However, their functions are limited due to dedicated design. In recent years, Supernumerary Robotic Limbs (SRLs) have become a hot spot in the field of wearable robots. Different from exoskeletons and prostheses, SRLs compensate and strengthen human abilities by providing extra limbs. This advantage allows SRLs to assist users in a novel way, rather than substituting missing limbs or enhancing existing limbs. However, finding a trade-off between wearability, efficiency, and usability of those SRLs is still an issue. This paper presents the state of the art in SRLs and discusses some open questions about SRLs’ design and control for further research. This review covers the following areas: (1) Basic concepts and classifications of SRLs; (2) The literature retrieval methodology; (3) Design functions of different types of SRLs, including their positive and negative aspects; (4) Different control strategies of SRLs, including positive and negative aspects, and some improvement methods in applying SRLs; (5) The impact on human body schema while using SRLs; (6) Open challenges and suggestions for future development. This review will help researchers understand the current state of SRLs and provide comprehensive knowledge foundations for them.

\medskip
\subsubsection{Lacava, G., A. Marotta, F. Martinelli, A. Saracino, A. La Marra, E. Gil-Uriarte, and V. Mayoral Vilches. Current research issues on cyber security in robotics. Technical report, 2020.}
see \cite{lacava2020current}

Cyber Security in Robotics is a rapidly developing area which draws attention from
practitioners and researchers. In this paper we provided an overview of the key issues
arising in the cyber security robotic landscape and the threats affecting this sector. We
also analyzed the scientific approaches to managing cyber attacks in robotics. Finally, we
proposed directions for further advances in this area

\medskip
\subsubsection{Farokh Atashzar, S., Mahdi Tavakoli, Dario Farina, and Rajni V. Patel. "Autonomy and Intelligence in Neurorehabilitation Robotic and Prosthetic Technologies." (2020): 2002001.}
see \cite{farokh2020autonomy}

Neurorehabilitation robotic technologies and powered assistive prosthetic devices have shown great potential for accelerating motor recovery or compensating for the lost motor functions of disabled users. The functioning of these technologies relies on a highly-interactive bidirectional flow of information and physical energy between a human user and a robotic system. Thus, key factors are integrity, intelligence and quality of the interaction loops. As a result, research in this field has focused on (a) enhancing the quality and safety of the physical interaction between disabled users and robotic systems while providing a high level of intelligence and adaptability for generating assistive and therapeutic force fields; (b) detecting the user’s motor intention with high spatiotemporal resolution to provide bidirectional human–machine interfacing; (c) promoting mental engagement through designing multimodal interactive interfaces and various sensory manipulation strategies. This Special Issue has collected papers that contribute to these three research areas, highlighting the importance of different aspects in human–robot interaction loops for augmenting the performance of neurorehabilitation robotic systems and prosthetic devices.

\medskip
\subsubsection{Oetgen, Matthew E., Jody Litrenta, Bamshad Azizi Koutenaei, and Kevin R. Cleary. "A novel surgical navigation technology for placement of implants in slipped capital femoral epiphysis." The International Journal of Medical Robotics and Computer Assisted Surgery 16, no. 1 (2020): e2070.}
see \cite{oetgen2020novel}

Background

Fixation with a single screw is the recommended treatment for slipped capital femoral epiphysis (SCFE). Achieving optimal implant positioning can be difficult owing to the complex geometry of the proximal femur in SCFE. We assessed a novel navigation technology incorporating an inertial measurement unit to facilitate implant placement in an SCFE model.

Methods

Guidewires were placed into 30 SCFE models, using a navigation system that displayed the surgeon's projected implant trajectory simultaneously in multiple planes. The accuracy and the precision of the system were assessed as was the time to perform the procedure.

Results

Implants were placed an average of 5.3mm from the femoral head center, with a system precision of 0.94mm. The actual trajectory of the implant deviated from the planned trajectory by an average of 4.9°±2.2°. The total average procedure time was 97 seconds.

Conclusion

The use of computer-based navigation in a SCFE model demonstrated good accuracy and precision in terms of both implant trajectory and placement in the center of the femoral head.

\medskip
\subsubsection{Tolu, Gheorghe, Daniel Ghiculescu, and Miron Zapciu. "THE NONCONVENTIONAL SURGICAL SYSTEM DA VINCI." Revista de Tehnologii Neconventionale 24, no. 1 (2020): 39-43.}
see \cite{tolu2020nonconventional}

Intuitive Surgical is the pioneer and a global technology leader in robotic-assisted, minimally invasive surgery. The company develops, manufactures and markets the Surgical System da Vinci, the most complex device used in medical robotics. The product is called "da Vinci" because Leonardo da Vinci invented the first robot, and his works excel in anatomical details. The Surgical System da Vinci is a robotic platform that allows complex surgery through incisions of 1-2 cm. So far, hundreds of thousands of surgeries have been performed. The Surgical System da Vinci reproduces the surgeon's movements in real time. The advantages of using the Surgical System da Vinci, are greater precision, a much better picture and remote surgery. The disadvantages to traditional techniques are that it cannot be programmed and cannot make decisions on its own to make a surgical move or surgery without the surgeon's command. The Surgical System da Vinci is used currently in major medical centers around the world.

\medskip
\subsubsection{Martinez, Daniel Enrique, Waiman Meinhold, John Oshinski, Ai-Ping Hu, and Jun Ueda. "Resolution-Enhanced MRI-Guided Navigation of Spinal Cellular Injection Robot." In 2020 International Symposium on Medical Robotics (ISMR), pp. 83-88. IEEE, 2020.}
see \cite{martinez2020resolution}

This paper presents a method of navigating a surgical robot beyond the resolution of magnetic resonance imaging (MRI) by using a resolution enhancement technique enabled by high-precision piezoelectric actuation. The surgical robot was specifically designed for injecting stem cells into the spinal cord. This particular therapy can be performed in a shorter time by using a MRI-compatible robotic platform than by using a manual needle positioning platform. Imaging resolution of fiducial markers attached to the needle guide tubing was enhanced by reconstructing a high-resolution image from multiple images with sub-pixel movements of the robot. The parallel-plane direct-drive needle positioning mechanism positioned the needle guide with a high spatial precision that is two orders of magnitude higher than typical MRI resolution up to 1 mm. Reconstructed resolution enhanced images were used to navigate the robot precisely that would not have been possible by using standard MRI. Experiments were conducted to verify the effectiveness of the proposed enhanced-resolution image-guided intervention.

\medskip
\subsubsection{Uslu, Tuğrul, Erkin Gezgin, Seda Özbek, Didem Güzin, Fatih Cemal Can, and Levent Çetin. "Utilization of Low Cost Motion Capture Cameras for Virtual Navigation Procedures: Performance Evaluation for Surgical Navigation." Measurement (2021): 109624.uslu2021utilization}
see \cite{uslu2021utilization}

Thanks to recent advances in medical robotics, various traditional surgical procedures have been started to be carried out by the help of robot manipulators. In order to enhance visual feedback and operation efficiency in these scenarios, utilization of virtual navigation techniques with specialized hardware has become a widespread choice. On the other hand, relatively high equipment costs have risks to slow down researches and mass adoption on the field. In light of this, current study represents performance evaluation of a low cost motion capture cameras on a scenario that tries to demonstrate robotic surgery like operation on a target patient through virtual navigation. Throughout the study least squares point based registration technique was utilized to correlate different reference frames with each other. A new approach was proposed for the calibration between robot manipulator and motion capture system to allow operation without using markers on manipulator side. Innovative patient mockup design with precisely formed landmark points was also introduced in order to verify performances of utilized low cost hardware. At the end of the study, hardware verification results showed the possibility of sub millimeter precisions in demonstrated navigation procedures.

\medskip
\subsubsection{Mehrdad, Sarmad, Fei Liu, Minh Tu Pham, Arnaud Lelevé, and S. Farokh Atashzar. "Review of advanced medical telerobots." Applied Sciences 11, no. 1 (2021): 209.}
see \cite{mehrdad2021review}

The advent of telerobotic systems has revolutionized various aspects of the industry and human life. This technology is designed to augment human sensorimotor capabilities to extend them beyond natural competence. Classic examples are space and underwater applications when distance and access are the two major physical barriers to be combated with this technology. In modern examples, telerobotic systems have been used in several clinical applications, including teleoperated surgery and telerehabilitation. In this regard, there has been a significant amount of research and development due to the major benefits in terms of medical outcomes. Recently telerobotic systems are combined with advanced artificial intelligence modules to better share the agency with the operator and open new doors of medical automation. In this review paper, we have provided a comprehensive analysis of the literature considering various topologies of telerobotic systems in the medical domain while shedding light on different levels of autonomy for this technology, starting from direct control, going up to command-tracking autonomous telerobots. Existing challenges, including instrumentation, transparency, autonomy, stochastic communication delays, and stability, in addition to the current direction of research related to benefit in telemedicine and medical automation, and future vision of this technology, are discussed in this review paper.

\medskip
\subsubsection{Liu, Yajun, Peihao Jin, Wenyong Liu, and Wei Tian. "Basic Principle of Robot-Assisted Orthopedic Surgery." In Navigation Assisted Robotics in Spine and Trauma Surgery, pp. 5-10. Springer, Singapore, 2020.}
see \cite{liu2020basic}

The collaboration between robot and medical environment (including medical staff) plays a critical role through the entire procedure of robot-assisted orthopedic surgery. From aspects of surgical informatization and interactivity, this chapter introduces the functional configuration (workflow and basic setup) and the human–robot interaction modes in the orthopedic operating room. Suggestions for improving the performance and the clinical acceptability of the robot system is also briefly discussed.

\medskip
\subsubsection{Yang, Jianxing, Yan Xiong, Xiaohong Chen, Yuanxi Sun, Wensheng Hou, Rui Chen, Shandeng Huang, and Long Bai. "Bone Fracture Reduction Surgery-aimed Bone Connection Robotic Hand." Journal of Bionic Engineering 18, no. 2 (2021): 333-345.}
see \cite{yang2021bone}

Bone connection with robot is an important topic in the research of robot assisted fracture reduction surgery. With the method to achieve bone-robot connection in current robots, requirements on reliability and low trauma can not be satisfied at the same time. In this paper, the design, manufacturing, and experiments of a novel Bone Connection Robotic Hand (BCRH) with variable stiffness capability are carried out through the bionics research on human hand and the principle of particle jamming. BCRH’s variable stiffness characteristic is a special connection between “hard connection” and “soft connection”, which is different from the existing researches. It maximizes the reliability of bone-robot connection while minimizes trauma, meets the axial load requirement in clinical practice, and effectively shortens the operating time to less than 40 s (for mode 1) or 2 min (for mode 2). Meanwhile, a theoretical analysis of bone-robot connection failure based on particle jamming is carried out to provide references for the research in this paper and other related studies

\medskip
\subsubsection{Zhang, Wu, Haiyuan Li, Linlin Cui, Haiyang Li, Xiangyan Zhang, Shanxiang Fang, and Qinjian Zhang. "Research progress and development trend of surgical robot and surgical instrument arm." The International Journal of Medical Robotics and Computer Assisted Surgery (2021): e2309.}
see \cite{zhang2021research}

Background

In recent years, surgical robots have become an indispensable part of the medical field. Surgical robots are increasingly being used in the areas of gynaecological surgery, urological surgery, orthopaedic surgery, general surgery and so forth. In this paper, the development of surgical robots in different operations is reviewed and analysed. In the type of master–slave surgical robotic system, the robotic surgical instrument arms were located in the execution terminal of a surgical robot system, as one of the core components, and directly contact with the patient during the operation, which plays an important role in the efficiency and safety of the operation. In clinical, the arm function and design in different systems varies. Furtherly, the current research progress of robotic surgical instrument arms used in different operations is analysed and summarised. Finally, the challenge and trend are concluded.

Methods

According to the classification of surgical types, the development of surgical robots for laparoscopic surgery, neurosurgery, orthopaedics and microsurgery are analysed and summarised. Then, focusing on the research of robotic surgical instrument arms, according to structure type, the research and application of straight-rod surgical instrument arm, joint surgical instrument arm and continuous surgical instrument arm are analysed respectively.

Results

According to the discussion and summary of the characteristics of the existing surgical robots and instrument arms, it is concluded that they still have a lot of room for development in the future. Therefore, the development trends of the surgical robot and instrument arm are discussed and analysed in the five aspects of structural materials, modularisation, telemedicine, intelligence and human–machine collaboration.

Conclusion

Surgical robots have shown the development trend of miniaturisation, intelligence, autonomy and dexterity. Thereby, in the field of science and technology, the research on the next generation of minimally invasive surgical robots will usher in a peak period of development.

\medskip
\subsubsection{Yang, Geng, Zhibo Pang, M. Jamal Deen, Mianxiong Dong, Yuan-Ting Zhang, Nigel Lovell, and Amir M. Rahmani. "Homecare robotic systems for healthcare 4.0: visions and enabling technologies." IEEE journal of biomedical and health informatics 24, no. 9 (2020): 2535-2549.}
see \cite{yang2020homecare}

Powered by the technologies that have originated from manufacturing, the fourth revolution of healthcare technologies is happening (Healthcare 4.0). As an example of such revolution, new generation homecare robotic systems (HRS) based on the cyber-physical systems (CPS) with higher speed and more intelligent execution are emerging. In this article, the new visions and features of the CPS-based HRS are proposed. The latest progress in related enabling technologies is reviewed, including artificial intelligence, sensing fundamentals, materials and machines, cloud computing and communication, as well as motion capture and mapping. Finally, the future perspectives of the CPS-based HRS and the technical challenges faced in each technical area are discussed.

\medskip
\subsubsection{Zhang, Zhuangzhuang, Qixin Cao, Xiaoxiao Zhu, Yiqi Yang, and Nan Luan. "External Force Estimation on a Robotic Surgical Instrument." In 2020 5th International Conference on Advanced Robotics and Mechatronics (ICARM), pp. 263-268. IEEE, 2020.}
see \cite{zhang2020external}

In this paper, a novel force/torque estimation algorithm for the in-house developed instrument in the robotic-assisted arthroscopic surgery system is proposed. This surgical robot system consists of two parts with 7 degree-of-freedom (DOF) Franka Emika robot for providing 4-DOF Remote Centre of Motion (RCM) about the incision-trocar and an instrument performing bone grinding. The method utilizes Neural Networks (NN) in the Cartesian space to estimate external forces acting on the instrument. The instrument is a rigid-link mechanism attached to the end of the Franka robot by a 6-DOF wrist force sensor. With this proposed method it is possible to obtain force and torque estimation in Cartesian space for any rigid-link wrist mechanism under RCM constraints. Several experiments are performed on an actual robotic system prototype and results show the efficacy of the proposed method.

\medskip
\subsubsection{Bhandari, Mahendra, Trevor Zeffiro, and Madhu Reddiboina. "Artificial intelligence and robotic surgery: current perspective and future directions." Current opinion in urology 30, no. 1 (2020): 48-54.}
see \cite{bhandari2020artificial}

Purpose of review 

This review aims to draw a road-map to the use of artificial intelligence in an era of robotic surgery and highlight the challenges inherent to this process.

Recent findings 

Conventional mechanical robots function by transmitting actions of the surgeon's hands to the surgical target through the tremor-filtered movements of surgical instruments. Similarly, the next iteration of surgical robots conform human-initiated actions to a personalized surgical plan leveraging 3D digital segmentation generated prior to surgery. The advancements in cloud computing, big data analytics, and artificial intelligence have led to increased research and development of intelligent robots in all walks of human life. Inspired by the successful application of deep learning, several surgical companies are joining hands with tech giants to develop intelligent surgical robots. We, hereby, highlight key steps in the handling and analysis of big data to build, define, and deploy deep-learning models for building autonomous robots.

Summary 

Despite tremendous growth of autonomous robotics, their entry into the operating room remains elusive. It is time that surgeons actively collaborate for the development of the next generation of intelligent robotic surgery.

\medskip
\subsubsection{Prokhorenko, Leonid, Daniil Klimov, Denis Mishchenkov, and Yuri Poduraev. "Surgeon–robot interface development framework." Computers in biology and medicine 120 (2020): 103717.}
see \cite{prokhorenko2020surgeon}

The progress of robotic medicine leads to the emergence of an increasing number of highly specialized automated systems based on specialized software. In any such system, there is the task of translating the surgeon’s requests into the process of automated procedure execution. The hardware and software system that provides the translation is the interface between the surgeon and the robot. This paper proposes a generalized framework architecture for the development of such software — the surgeon–robot interface. Existing implementations of such an interface are considered, solutions for the internal structure design of the framework are proposed. Experiments were performed using a prototype of the proposed framework. Such a development framework will allow one to effectively implement the surgeon–robot interfaces at all stages of the robotization of medical procedures, from prototype to final use in the operating room.

\medskip
\subsubsection{Song, Mi Ok, and Yong Jin Cho. "The Present and Future of Medical Robots: Focused on Surgical Robots." Journal of Digital Convergence 19, no. 4 (2021): 349-353.}
see \cite{song2021present}

This study is a review study attempted to analyze the current situation of surgical robots based on previous research on surgical robots in the era of the 4th revolution, and to forecast the future direction of surgical robots. Surgical robots have made full progress since the launch of the da Vinci and the surgical robot is playing a role of supporting the surgeries of the surgeons or the master-slave method reflecting the intention of the surgeons. Recently, technologies are being developed to combine artificial intelligence and big data with surgical robots, and to commercialize a universal platform rather than a platform dedicated to surgery. Moreover, technologies for automating surgical robots are being developed by generating 3D image data based on diagnostic image data, providing real-time images, and integrating image data into one system. For the development of surgical robots, cooperation with clinicians and engineers, safety management of surgical robot, and institutional support for the use of surgical robots will be required.

\medskip
\subsubsection{Kadkhodamohammadi, Abdolrahim, Nachappa Sivanesan Uthraraj, Petros Giataganas, Gauthier Gras, Karen Kerr, Imanol Luengo, Sam Oussedik, and Danail Stoyanov. "Towards video-based surgical workflow understanding in open orthopaedic surgery." Computer Methods in Biomechanics and Biomedical Engineering: Imaging \& Visualization (2020): 1-8.}
see \cite{kadkhodamohammadi2020towards}

Safe and efficient surgical training and workflow management play a critical role in clinical competency and ultimately, patient outcomes. Video data in minimally invasive surgery (MIS) have enabled opportunities for vision-based artificial intelligence (AI) systems to improve surgical skills training and assurance through post-operative video analysis and development of real-time computer-assisted interventions (CAI). Despite the availability of mounted cameras for the operating room (OR), similar capabilities are much more complex to develop for recording open surgery procedures, which has resulted in a shortage of exemplar video-based training materials. In this paper, we present a potential solution to record open surgical procedures using head-mounted cameras. Recorded videos were anonymised to remove patient and staff identifiable information using a machine learning algorithm that achieves state-of-the-art results on the OR Face dataset. We then propose a CNN-LSTM-based model to automatically segment videos into different surgical phases, which has never been previously demonstrated in open procedures. The redacted videos, along with the automatically predicted phases, are then available for surgeons and their teams for post-operative review and analysis. To our knowledge, this is the first demonstration of the feasibility of deploying camera recording systems and developing machine learning-based workflow analysis solutions for open surgery, particularly in orthopaedics.

\medskip
\subsubsection{Li, Changsheng, Xiaoyi Gu, Xiao Xiao, Chwee Ming Lim, Xingguang Duan, and Hongliang Ren. "A flexible transoral robot towards covid-19 swab sampling." Frontiers in Robotics and AI 8 (2021): 51.}
see \cite{li2021flexible}

There are high risks of infection for surgeons during the face-to-face COVID-19 swab sampling due to the novel coronavirus’s infectivity. To address this issue, we propose a flexible transoral robot with a teleoperated configuration for swab sampling. The robot comprises a flexible manipulator, an endoscope with a monitor, and a master device. A 3- prismatic-universal (3-PU) flexible parallel mechanism with 3 degrees of freedom (DOF) is used to realize the manipulator’s movements. The flexibility of the manipulator improves the safety of testees. Besides, the master device is similar to the manipulator in structure. It is easy to use for operators. Under the guidance of the vision from the endoscope, the surgeon can operate the master device to control the swab’s motion attached to the manipulator for sampling. In this paper, the robotic system, the workspace, and the operation procedure are described in detail. The tongue depressor, which is used to prevent the tongue’s interference during the sampling, is also tested. The accuracy of the manipulator under visual guidance is validated intuitively. Finally, the experiment on a human phantom is conducted to demonstrate the feasibility of the robot preliminarily.

\medskip
\subsubsection{Ohnishi, Ayumi, Hayate Tohnan, Tsutomu Terada, Minoru Hattori, Hisaaki Yoshinaka, Yusuke Sumi, Hiroyuki Egi, and Masahiko Tsukamoto. "A Method for Estimating Doctor's Fatigue Level in Operating a Surgical Robot Using Wearable Sensors." In 2021 IEEE International Conference on Pervasive Computing and Communications Workshops and other Affiliated Events (PerCom Workshops), pp. 38-43. IEEE, 2021.}
see \cite{ohnishi2021method}

Robot-assisted laparoscopic surgery, such as the da Vinci Surgical System, has a problem in that surgeons might continue operating for a long period of time without realizing their fatigue and without proper concentration, or they might ignore their fatigue even when they notice it. We propose a method for quantitatively estimating the level fatigue by attaching wearable sensors to a doctor while using a surgical robot. In this study, several sensor configurations were tested to investigate the sensor configuration, which is easy for the surgeon to wear. In an evaluation experiment, doctors used a robotic surgery simulator for a long period of time, and the score calculated by the simulator was estimated as the fatigue level expressed by the surgeon. We discussed with doctors how the results of the fatigue estimation should be applied to the system design.

\medskip
\subsubsection{Bondarenko, Viktor, Andrey Kholyavin, Yaroslav Belyaev, Dmitry Epifanov, Islam Bzhikhatlov, Mikhail Abramchuk, and Maxim Mokeyev. "Development of a 6-axis Robotic Manipulator for Stereotactic Surgery." In 2020 XI International Conference on Electrical Power Drive Systems (ICEPDS), pp. 1-4. IEEE, 2020.}
see \cite{bondarenko2020development}

The paper presents the results of development of a 6-axis robotic manipulator for stereotactic surgery. The robotic manipulator can be used for the surgical treatment of deep-located brain tumors, for the intracerebral implantation of electrodes for brain stimulation in patients with Parkinson Disease, for the treatment of epilepsy, etc. The experimental model of the low-cost robotic manipulator is described in details. Basic requirements for stereotactic robotic manipulator are as follows: workspace is not less than 0.4 m3, position accuracy is ± 0.5 mm, the ability to change the angle of the stereotactic instrument while the coordinates of its end point are the same, compatibility with the surgical navigation systems, and compliance with the requirements for medical equipment. The results of a phantom testing of the manipulator in the neurosurgery operation room using the surgical optical navigation system showed that it is possible to use this manipulator for all types of stereotactic surgery interventions on the brain.

\medskip
\subsubsection{Tian, Wei, Yi Wei, and Xiaoguang Han. "The history and development of robot-assisted orthopedic surgery." In Navigation Assisted Robotics in Spine and Trauma Surgery, pp. 1-3. Springer, Singapore, 2020.}
see \cite{tian2020history}

Orthopedic surgical robot is the core intelligent equipment to promote the development of precision, minimally invasive orthopedics surgery, which has become the focus of international researches. This chapter introduces the development of orthopedic surgical robots and the typical products of orthopedic robots.

\medskip
\subsubsection{Omisore, Olatunji Mumini, Shipeng Han, Jing Xiong, Hui Li, Zheng Li, and Lei Wang. "A review on flexible robotic systems for minimally invasive surgery." IEEE Transactions on Systems, Man, and Cybernetics: Systems (2020).}
see \cite{omisore2020review}

Recently, flexible robotic systems are developed to enhance minimally invasive interventions on internal organs located in confined areas of human body. These surgical devices are designed to navigate anatomical pathways via single-port access, such as natural orifices or minimal incisions and intraluminal interventions. With improved precision, spatial flexibility and dexterity, the robotic technology can enhance surgery such that minimally invasive flexible access would become a faster, safer, and more convenient method for intra-body interventions without multiple or wide incisions. However, a lot of works are still required for global acceptance of existing flexible robotic surgical platforms. This review provides extended insights on the design details of two types of flexible robotic systems used for endoscopic and endovascular procedures. As of today, several prototypes of both platforms have been proposed; however, their global acceptability and applicability remains very low. To address these, we present an extensive review on design constraints and control methods which are vital for safer, faster, and better operation of the flexible robotic systems in minimally invasive surgery (MIS). Finally, research trends of flexible robotic systems and their clinical application status in MIS are discussed along with some of the technical and technological challenges hindering their prominence.

\medskip
\subsubsection{Zemmar, Ajmal, Andres M. Lozano, and Bradley J. Nelson. "The rise of robots in surgical environments during COVID-19." Nature Machine Intelligence 2, no. 10 (2020): 566-572.}
see \cite{zemmar2020rise}

The COVID-19 pandemic has changed our world and impacted multiple layers of our society. All frontline workers and in particular those in direct contact with patients have been exposed to major risk. To mitigate pathogen spread and protect healthcare workers and patients, medical services have been largely restricted, including cancellation of elective surgeries, which has posed a substantial burden for patients and immense economic loss for various hospitals. The integration of a robot as a shielding layer, physically separating the healthcare worker and patient, is a powerful tool to combat the omnipresent fear of pathogen contamination and maintain surgical volumes. In this Perspective, we outline detailed scenarios in the pre-, intra- and postoperative care, in which the use of robots and artificial intelligence can mitigate infectious contamination and aid patient management in the surgical environment during times of immense patient influx. We also discuss cost-effectiveness and benefits of surgical robotic systems beyond their use in pandemics. The current pandemic creates unprecedented demands for hospitals. Digitization and machine intelligence are gaining significance in healthcare to combat the virus. Their legacy may well outlast the pandemic and revolutionize surgical performance and management.

\medskip
\subsubsection{Cheng, Ching-Hwa. "A Real-Time Robot-Arm Surgical Guiding System Development by Image-Tracking." In 2020 2nd IEEE International Conference on Artificial Intelligence Circuits and Systems (AICAS), pp. 133-133. IEEE, 2020.}
see \cite{cheng2020real}

The endoscope is widely used for various diagnoses and treatments in Minimally Invasive Surgery (MIS), such as hysteroscopy, laparoscopy, and colonoscopy. However, the limited field of image of the endoscope is often the most problematic issue faced by surgeons and medical students, especially for those inexperienced physicians, which leads to difficulty during surgical operations. To reduce the difficulties of MIS with respect to endoscope function, the proposed identifying and locating techniques provide the angle and distance from the surgical instruments to the lesion. The in-time guiding information provides global positioning information by tracking the lesion position during surgery. The jointed with a robot-arm system can help an inexperienced surgeon with stable assistance for the long-time surgical operation. The whole system has been successfully validated by surgeons.

\medskip
\subsubsection{Roy, Rupanjan. "Medical Applications of Artificial Intelligence." (2021).}
see \cite{roy2021medical}

The medical \& the dental field is a never ending field of innovations \& developments and each time the reasearchers come up with something new. One such new dimension in the fields of medicine being the incorporation of Artificial intelligence assisted technologies improving diagnosis, treatmemt plan and treatment stategies. This review focusses on the application of different technologies of AI in different fields of medicine.

\medskip
\subsubsection{Su, Hang, Andrea Mariani, Salih Ertug Ovur, Arianna Menciassi, Giancarlo Ferrigno, and Elena De Momi. "Toward teaching by demonstration for robot-assisted minimally invasive surgery." IEEE Transactions on Automation Science and Engineering 18, no. 2 (2021): 484-494.}
see \cite{su2021toward}

Learning manipulation skills from open surgery provides more flexible access to the organ targets in the abdomen cavity and this could make the surgical robot working in a highly intelligent and friendly manner. Teaching by demonstration (TbD) is capable of transferring the manipulation skills from human to humanoid robots by employing active learning of multiple demonstrated tasks. This work aims to transfer motion skills from multiple human demonstrations in open surgery to robot manipulators in robot-assisted minimally invasive surgery (RA-MIS) by using TbD. However, the kinematic constraint should be respected during the performing of the learned skills by using a robot for minimally invasive surgery. In this article, we propose a novel methodology by integrating the cognitive learning techniques and the developed control techniques, allowing the robot to be highly intelligent to learn senior surgeons’ skills and to perform the learned surgical operations in semiautonomous surgery in the future. Finally, experiments are performed to verify the efficiency of the proposed strategy, and the results demonstrate the ability of the system to transfer human manipulation skills to a robot in RA-MIS and also shows that the remote center of motion (RCM) constraint can be guaranteed simultaneously. Note to Practitioners —This article is inspired by limited access to the manipulation of laparoscopic surgery under a kinematic constraint at the point of incision. Current commercial surgical robots are mostly operated by teleoperation, which is representing less autonomy on surgery. Assisting and enhancing the surgeon’s performance by increasing the autonomy of surgical robots has fundamental importance. The technique of teaching by demonstration (TbD) is capable of transferring the manipulation skills from human to humanoid robots by employing active learning of multiple demonstrated tasks. With the improved ability to interact with humans, such as flexibility and compliance, the new generation of serial robots becomes more and more popular in nonclinical research. Thus, advanced control strategies are required by integrating cognitive functions and learning techniques into the processes of surgical operation between robots, surgeon, and minimally invasive surgery (MIS). In this article, we propose a novel methodology to model the manipulation skill from multiple demonstrations and execute the learned operations in robot-assisted minimally invasive surgery (RA-MIS) by using a decoupled controller to respect the remote center of motion (RCM) constraint exploiting the redundancy of the robot. The developed control scheme has the following functionalities: 1) it enables the 3-D manipulation skill modeling after multiple demonstrations of the surgical tasks in open surgery by integrating dynamic time warping (DTW) and Gaussian mixture model (GMM)-based dynamic movement primitive (DMP) and 2) it maintains the RCM constraint in a smaller safe area while performing the learned operation in RA-MIS. The developed control strategy can also be potentially used in other industrial applications with a similar scenario

\medskip
\subsubsection{Phan, Gia Hoang. "Humanoid robotics in healthcare: A review." Design Engineering (2021): 3641-3656.
}
see \cite{phan2021humanoid}

This page seeks to help scientists and the wider community better think that makes a robot pleasant by giving an overview of healthcare robot ideas, laboratory testing, and applications. When healthcare robots are utilized appropriately for their structure and functions, they show their capabilities. Companions for the elderly and others with cognitive impairments, robots in educational settings, and cognitive and behavioral enhancement technology are just a few examples. While the robots shown in films and literature remain futuristic, science fiction has inspired everybody to envision a world in which robotics help us in every aspect of our everyday lives. While we have a long way to go before robots are ubiquitous in our social spaces, significant advances in healthcare robotics technology, supported by social sciences, are bringing us closer.

\medskip
\subsubsection{Gasteiger, Norina, and Elizabeth Broadbent. "AI, robotics, medicine and health sciences." In The Routledge Social Science Handbook of AI, pp. 313-338. Routledge, 2021.}
see \cite{gasteiger2021ai}

The emergence of artificial intelligence (AI) has provided many opportunities for improvements in healthcare. Early expectations were for AI to change the role of physicians, recruitment and education. This chapter explores the historical and intellectual development of AI and robotics, with a focus on health purposes. It covers major claims and developments, principal contributions to healthcare and major criticisms of using AI and robots in healthcare. Neural networks and Deep learning are complex techniques of machine learning. Artificial neural networks vaguely simulate how neurons in the brain would process signals. The 1980s and 1990s were characterised by a surge of interest in AI, especially the application of neural networks, fuzzy set theory and Bayesian networks. Wearable computing refers to computer-powered wearable items, such as clothing, earphones, shoes, socks, watches, wristbands and glasses. Mimicking the success and acceptance of animal therapy, many companion robots look and behave like real animals.

\medskip
\subsubsection{Sadiku, Matthew NO, Nishu Gupta, Yogita P. Akhare, and Sarhan M. Musa. "Emerging IoT Technologies in Smart Healthcare." In IoT and ICT for Healthcare Applications, pp. 3-10. Springer, Cham, 2020.}
see \cite{sadiku2020emerging}

The digital revolution seeks to transform healthcare and empower citizens in taking charge of their own health. Healthcare services amount to a considerable portion of the world economy. Therefore, we need to have a check over its rising costs, which is a major concern for any nation. The advancement of healthcare has been attributed to innovation of new medical devices and technologies. The new devices provide innovative solutions for diagnosis, prevention, and treatments. This chapter begins by discussing the concept of emerging technology. Then, it covers several emerging technologies in healthcare. It discusses some of the applications of the emerging technologies. It presents some the benefits and challenges of the emerging technologies. The last section concludes with some comments.

\medskip
\subsubsection{Khalid, Sibar. "Internet of Robotic Things: A Review." Journal of Applied Science and Technology Trends 2, no. 03 (2021): 78-90.}
see \cite{khalid2021internet}

The  Internet  of  Things  (IoT)  gives  a strong  structure  for  connecting  things  to  the  internet  to  facilitate  Machine  to  Machine  (M2M) communication and data transmission through basic network protocols such as TCP/IP.  IoT is growing at a fast pace, and billions of devices are now associated, with the amount expected to reach trillions in the coming years. Many fields, including the army, farming, manufacturing,  healthcare,  robotics,  and  biotechnology,  are  adopting  IoT  for  advanced  solutions  as  technology  advances.  This paper offers a detailed view of the current IoT paradigm, specifically proposed for robots, namely the Internet of Robotic Things (IoRT). IoRT is a collection of various developments such as Cloud Computing, Artificial Intelligence (AI), Machine Learning, and the (IoT). This paper also goes over architecture, which would be essential in the design of Multi-Role Robotic Systems for IoRT. Furthermore, includes systems underlying IoRT, as well as IoRT implementations.  The paper provides the foundation for researchers to imagine the idea of IoRT and to look beyond the frame while designing and implementing IoRT-based robotic systems in real-world implementations.

\medskip
\subsubsection{Boubaker, Olfa. "Medical robotics." Control Theory in Biomedical Engineering (2020): 153-204.}
see \cite{boubaker2020medical}

Today, robotic devices are used for delicate surgical procedures from open surgery to minimally invasive procedures, replacing missing limbs, delivering neuro-rehabilitation therapy to stroke patients, teaching people with learning disabilities, administering drugs, and performing a growing number of other health tasks. This chapter highlights the impact of these machines to improve efficiency and precision of human abilities. Through a comprehensive review of the literature, this chapter presents the different classification approaches of robotic devices. More than 150 references in the open literature, most of them survey papers, are compiled to provide a historical point of view in medical robotics, a review in emerging robotic systems, and an investigation in related challenging problems.

\medskip
\subsubsection{Cooper, Sara, Alessandro Di Fava, Carlos Vivas, Luca Marchionni, and Francesco Ferro. "ARI: The social assistive robot and companion." In 2020 29th IEEE International Conference on Robot and Human Interactive Communication (RO-MAN), pp. 745-751. IEEE, 2020.}
see \cite{cooper2020ari}

With the world population aging and the number of healthcare users with multiple chronic diseases increasing, healthcare is becoming more costly, and as such, the need to optimise both hospital and in-home care is of paramount importance. This paper reviews the challenges that the older people, people with mobility constraints, hospital patients and isolated healthcare users face, and how socially assistive robots can be used to help them. Related promising areas and limitations are highlighted. The main focus is placed on the newest PAL Robotics' robot: ARI, a high-performance social robot and companion designed for a wide range of multi-modal expressive gestures, gaze and personalised behaviour, with great potential to become part of the healthcare community by applying powerful AI algorithms. ARI can be used to help administer first-care attention, providing emotional support to people who live in isolation, including the elderly population or healthcare users who are confined because of infectious diseases such as Covid-19. The ARI robot technical features and potential applications are introduced in this paper.


\medskip
\subsubsection{Saniotis, Arthur, and Maciej Henneberg. "Neurosurgical robots and ethical challenges to medicine." Ethics in Science and Environmental Politics 21 (2021): 25-30.
}
see \cite{saniotis2021neurosurgical}

Over the last 20 yr, neurosurgical robots have been increasingly assisting in neurosurgical procedures. Surgical robots are considered to have noticeable advantages over humans, such as reduction of procedure time, surgical dexterity, no experience of fatigue and improved healthcare outcomes. In recent years, neurosurgical robots have been developed to perform various procedures. Public demand is informing the direction of neurosurgery and placing greater pressure on neurosurgeons to use neurosurgical robots. The increasing diversity and sophistication of neurosurgical robots have received ethical scrutiny due to the surgical complications that may arise as well as the role of robots in the future. In this paper, we address 3 ethical areas regarding neurosurgical robots: (1) Loss of neurosurgical skills due to increasing dependency on robots; (2) How far do we want to go with neurosurgical robots? (3) Neurosurgical robots and conflict of interest and medical bias.

\medskip
\subsubsection{Fischer, Kerstin, Johanna Seibt, Raffaele Rodogno, Maike Kirkegård Rasmussen, Astrid Weiss, Leon Bodenhagen, William Kristian Juel, and Norbert Krüger. "Integrative social robotics hands-on." Interaction Studies 21, no. 1 (2020): 145-185.}
see \cite{fischer2020integrative}

In this paper, we discuss the development of robot use cases in an elderly care facility in the context of exploring the method of Integrative Social Robotics (ISR) when used on top of a user-centered design approach. Integrative Social Robotics is a new proposal for how to generate responsible, i.e. culturally and ethically sustainable, social robotics applications. Starting point for the discussion are the five principles that characterize an ISR approach, which are discussed in application to the three use cases for robot support in a Danish elderly care facility developed within the SMOOTH project. The discussion by an interdisciplinary design team explores what attention to the five principles of ISR can offer for use case development. We report on the consequences of this short-time exposure to the basic ideas of ISR for use case development and discuss the value of approaching robot development from an ISR perspective.

\medskip
\subsubsection{Seidita, Valeria, Francesco Lanza, Arianna Pipitone, and Antonio Chella. "Robots as intelligent assistants to face COVID-19 pandemic." Briefings in Bioinformatics 22, no. 2 (2021): 823-831.}
see \cite{seidita2021robots}

Motivation

The epidemic at the beginning of this year, due to a new virus in the coronavirus family, is causing many deaths and is bringing the world economy to its knees. Moreover, situations of this kind are historically cyclical. The symptoms and treatment of infected patients are, for better or worse even for new viruses, always the same: more or less severe flu symptoms, isolation and full hygiene. By now man has learned how to manage epidemic situations, but deaths and negative effects continue to occur. What about technology? What effect has the actual technological progress we have achieved? In this review, we wonder about the role of robotics in the fight against COVID. It presents the analysis of scientific articles, industrial initiatives and project calls for applications from March to now highlighting how much robotics was ready to face this situation, what is expected from robots and what remains to do.

Results

The analysis was made by focusing on what research groups offer as a means of support for therapies and prevention actions. We then reported some remarks on what we think is the state of maturity of robotics in dealing with situations like COVID-19.

\medskip
\subsubsection{Kubota, Alyssa, and Laurel D. Riek. "Methods for robot behavior adaptation for cognitive neurorehabilitation." Annual Review of Control, Robotics, and Autonomous Systems 5 (2021).}
see \cite{kubota2021methods}

An estimated 11\% of adults report experiencing some form of cognitive decline which may be associated with conditions such as stroke or dementia, and can impact their memory, cognition, behavior, and physical abilities. While there are no known pharmacological treatments for many of these conditions, behavioral treatments such as cognitive training can prolong the independence of people with cognitive impairments. These treatments teach metacognitive strategies to compensate for memory difficulties in their everyday lives. Personalizing these treatments to suit the preferences and goals of an individual is critical to improving their engagement and sustainment, as well as maximizing the treatment’s effectiveness. Robots have great potential to facilitate these training regimens and support people with cognitive impairments, their caregivers, and clinicians. This article examines how robots can adapt their behavior to be personalized to an individual in the context of cognitive neurorehabilitation. We provide an overview of existing robots being used to support neurorehabilitation, and identify key principles to working in this space. We then examine state-of-the-art technical approaches to enabling longitudinal behavioral adaptation. To conclude, we discuss our recent work on enabling social robots to automatically adapt their behavior and explore open challenges for longitudinal behavior adaptation. This work will help guide the robotics community as they continue to provide more engaging, effective, and personalized interactions between people and robots.

\medskip
\subsubsection{Roy, Ritam. "IMPLEMENTATION AND SENTIMENT ANALYSIS OF ARTIFICIAL INTELLIGENCE IN HEALTH CARE INDUSTRY." International Journal of Modern Agriculture 10, no. 2 (2021): 211-222.}
see \cite{roy2021implementation}

The implementation of Artificial Intelligence and Information Technology is taking a quantum leap in every industry. In recent years there has been an intensified focus on the utilization of Artificial Intelligence in different areas to take care of the complex issue. The same goes for the healthcare services industry. With the increase in complexity and rise of data in medical industry adaptation of Artificial Intelligence is growing at a rapid pace, it is bringing a paradigm shift to healthcare. The use of AI can enhance patient engagement and also through prediction it can help the hospitals better resource allocation. Although there are lots of opportunities for Artificial Intelligence in the health care sector but currently there are limited examples of such techniques being successfully deployed. This study aims to understand the consumer behaviour towards the adaptation of Artificial Intelligence in Health Care.

\medskip
\subsubsection{Virgos, Lucia Alonso, Miguel A. Sanchez Vidales, Fernando López Hernández, and J. Javier Rainer Granados. "Internet of Medical Things: Current and Future Trends." In Internet of Medical Things, pp. 19-36. CRC Press, 2021.}
see \cite{virgos2021internet}

The Internet of Medical Things (IoMT) establishes the necessary scenario for medical devices and applications to evolve through information technology. The ability to connect medical devices and systems increases the possibility of data storage, intelligently analyze data, interact, monitor and monitor with the user remotely, update the security concept.

This is a revolution in the field of medicine that allows there to be evident progress in its effectiveness. At present, there are certain implantation tendencies that seek, mainly, to offer innovations for improvement in the different areas. The future trend will be to offer standardized systems of implementation, improvement and evaluation of them, so that there may be a standardization that allows the medical sector to make a quantitative leap in quality.

In this chapter we analyze the most relevant current and future trends, referred to the IoMT. The objective is to offer a theoretical framework that allows to initiate new lines of research focused on offering new methods and/or analyzing carefully the benefits, the methods of implementation and the evaluation of each one of the existing systems.

\medskip
\subsubsection{Saini, Akanksha, A. J. Meitei, and Jitenkumar Singh. "Machine Learning in Healthcare: A Review." Available at SSRN 3834096 (2021).}
see \cite{saini2021machine}

This study attempts to introduce artificial intelligence and its significant subfields in machine learning algorithms and reviews the role of these subfields in various areas in healthcare such as bioinformatics, gene detection for cancer diagnosis, epileptic seizure, brain-computer interface. It also reviews the medical image processing through deep learning for diseases such as diabetic retinopathy, gastrointestinal disease, and tumour. And finally, this article discusses the real-world obstacles that need to be overcome to make AI techniques easier to use.

\medskip
\subsubsection{Miura, Satoshi, Ryutaro Ohta, Yang Cao, Kazuya Kawamura, Yo Kobayashi, and Masakatsu G. Fujie. "Using Operator Gaze Tracking to Design Wrist Mechanism for Surgical Robots." IEEE Transactions on Human-Machine Systems (2021).}
see \cite{miura2021using}

This article assessed how surgical robot parameters influenced operator viewpoint during a simulated surgical procedure. Surgical robots are useful tools in minimally invasive surgery. However, even with robots, suturing is difficult because the needle is sometimes obscured by tissue or manipulators and is thus not always visible during the procedure. This is especially true in pediatric surgery, where the surgical environment is smaller than in adult surgery. Hence, surgeons must carefully track the instruments and tissues to understand and predict their current and expected situations. In this article, we used gaze-tracking techniques to analyze the location and timing of the gaze of participants while they manipulated a virtual robotic surgical simulation system. To differentiate between the ideal and actual viewpoint trajectories, we conducted experiments with and without obstacles (i.e., simulated tissue and the manipulator arm). In the obstacle condition, we modulated the wrist length of the manipulator to bring it into view. In the no-obstacle condition, the participants mostly watched the suture needle tip. In the with-obstacle condition, the participants spent less time watching the instruments and more time watching the target point. The amount of time spent watching the target point increased as wrist length increased. Given this trade off relationship, we examined the proportion of time the participants spent looking at the instruments or target points by wrist length. We calculated the Pareto solutions and clarified the relationship between wrist length and the watching parts.

\medskip
\subsubsection{Miyachi, Shigeru, Yoshitaka Nagano, Reo Kawaguchi, Tomotaka Ohshima, and Hiroki Tadauchi. "Remote surgery using a neuroendovascular intervention support robot equipped with a sensing function: Experimental verification." Asian Journal of Neurosurgery 16, no. 2 (2021): 363.}
see \cite{miyachi2021remote}

Purpose: Expectations for remote surgery in endovascular treatments are increasing. We conducted the world's first remote catheter surgery experiment using an endovascular treatment-supported robot. We considered the results, examined the issues, and suggested countermeasures for practical use.

Methods: The slave robot in the angiography room is an original machine that enables sensing feedback by using an originally developed insertion force-measuring device, which detects the pressure stress on the vessel wall and alerts the operator using an audible scale. The master side was set in a separate room. They were connected via HTTP communication using local area network system. The surgeon operated by looking at a personal computer monitor that shared an angiography monitor. The slave robot catheterized and inserted a coil for an aneurysm in the silicon blood vessel model in the angiography room.

Results: Our robot responded to the surgeon's operations promptly and to the joystick's swift movements quite accurately. The surgeon could control the stress to the model vessels using various actions, because the operator could hear the sound from the insertion force. However, the robot required a time gradient to reach a stable advanced speed at the time of the initial movement, and experienced a slight time lag.

Conclusion: Our remote operation appeared to be sufficiently feasible to perform the surgery safely. This system seems extremely promising for preventing viral infection and radiation exposure to medical staff. It will also enable medical professionals to operate in remote areas and create a ubiquitous medical environment.

\medskip
\subsubsection{Cheng, Irene, Richard Moreau, Nathaniel Rossol, Arnaud Leleve, Patrick Lermusiux, Antoine Millon, and Anup Basu. "A Gesture-Based Interface for Remote Surgery." In Connected Health in Smart Cities, pp. 11-22. Springer, Cham, 2020.cheng2020gesture}
see \cite{cheng2020gesture}

There has been a great deal of research activity in computer- and robot-assisted surgeries in recent years. Some of the advances have included robotic hip surgery, image-guided endoscopic surgery, and the use of intra-operative MRI to assist in neurosurgery. However, most of the work in the literature assumes that all of the expert surgeons are physically present close to the location of a surgery. A new direction that is now worth investigating is assisting in performing surgeries remotely. As a first step in this direction, this chapter presents a system that can detect movement of hands and fingers, and thereby detect gestures, which can be used to control a catheter remotely. Our development is aimed at performing remote endovascular surgery by controlling the movement of a catheter through blood vessels. Our hand movement detection is facilitated by sensors, like LEAP, which can track the position of fingertips and the palm. In order to make the system robust to occlusions, we have improved the implementation by optimally integrating the input from two different sensors. Following this step, we identify high-level gestures, like push and turn, to enable remote catheter movements. To simulate a realistic environment we have fabricated a flexible endovascular mold, and also a phantom of the abdominal region with the endovascular mold integrated inside. A mechanical device that can remotely control a catheter based on movement primitives extracted from gestures has been built. Experimental results are shown demonstrating the accuracy of the system.

\medskip
\subsubsection{Mei, Ziyang. "Remote Vascular Interventional Surgery Robotics: A Review." (2021).}
see \cite{mei2021remote}

Interventional doctors are exposed to radiation hazards during the operation and endure high work intensity. Remote vascular interventional surgery robotics is a hot research field that can not only protect the health of interventional doctors, but also improve accuracy and efficiency of surgeries. However, the current vascular interventional robots still have many shortcomings to be improved. This article introduces the mechanical structure characteristics of various fields of vascular interventional therapy surgical robots, discusses the current key features of vascular interventional surgical robotics in force sensing, haptic feedback, and control methods, summarizes current frontiers about autonomous surgery, long geographic distances remote surgery and MRI-compatible structures. Finally, combined with the current research status of vascular interventional surgery robots, this article analyzes the development directions and puts forward a vision for the future vascular interventional surgery robots.

\medskip
\subsubsection{Legeza, Peter, Gavin W. Britz, Thomas Loh, and Alan Lumsden. "Current utilization and future directions of robotic-assisted endovascular surgery." Expert Review of Medical Devices 17, no. 9 (2020): 919-927.}
see \cite{legeza2020current}

Introduction

Endovascular surgery has become the standard of care to treat most vascular diseases using a minimally invasive approach. The CorPath system further enhances the potential and enables surgeons to perform robotic-assisted endovascular procedures in interventional cardiology, peripheral vascular surgery, and neurovascular surgery. With the introduction of this technique, the operator can perform multiple steps of endovascular interventions outside of the radiation field with high precision movements even from long-geographical distances.

Areas covered

The first and second-generation CorPath systems are currently the only commercially available robotic devices for endovascular surgery. This review article discusses the clinical experiences and outcomes with the robot, the advanced navigational features, and the results with recent hardware and software modifications, which enables the use of the system for neurovascular interventions, and long-distance interventional procedures.

Expert opinion

A high procedural success was achieved with the CorPath robotic systems in coronary and peripheral interventions, and the device seems promising in neurovascular procedures. More experience is needed with robotic neurovascular interventions and with complex peripheral arterial cases. In the future, long-distance endovascular surgery can potentially transform the management and treatment of acute myocardial infarction and stroke, with making endovascular care more accessible for patients in remote areas.

\medskip
\subsubsection{Arian, Y. "A Review of the Application of Robots in Maxillofacial Surgery." J Oral Health Dent Res 1, no. 1 (2021): 1-4.}
see \cite{arian2021review}

Aim: The purpose of this study was to review new articles on the use of robotic surgery in maxillofacial surgery.

Method and Materials: For the purpose of this review study, all Medline (PubMed), Google scholar electronic resources focused on the use of robotic surgery in maxillofacial surgery in the period 1999-2021 were reviewed.

Results: Using robots in maxillofacial surgery can reduce hospitalization time, reduce intraoperative bleeding, and improve recovery for patients, although the high cost and lack of touch can be a problem.

Conclusion: The results of this review study show that the surgery robot can replace open surgical methods of maxillofacial surgery. Although it may not be generalized for use, patients may be assisted in areas where the surgeon may not be present.

\medskip
\subsubsection{Desselle, Mathilde R., Ross A. Brown, Allan R. James, Mark J. Midwinter, Sean K. Powell, and Maria A. Woodruff. "Augmented and virtual reality in surgery." Computing in Science \& Engineering 22, no. 3 (2020): 18-26.}
see \cite{desselle2020augmented}

Augmented and virtual reality are transforming the practice of healthcare by providing powerful and intuitive methods of exploring and interacting with digital medical data, as well as integrating data into the physical world to create natural and interactive virtual experiences. These immersive technologies use lightweight stereoscopic head-mounted displays to place users into simulated and realistic three-dimensional digital environments, unlocking significant benefits from the seamless integration of digital information with the healthcare practitioner and patient's experience. This review article explores some of the current and emerging technologies and applications in surgery, their benefits and challenges around immersion, spatial awareness and cognition, and their reported and projected use in learning environments, procedure planning and perioperative contexts and in the surgical theatre. The enhanced access to information, knowledge, and experience enabled by virtual and augmented reality will improve healthcare approaches and lead to better outcomes for patients and the wider community.

\medskip
\subsubsection{Meshram, Dewanand A., and Dipti D. Patil. "5G Enabled Tactile Internet for Tele-Robotic Surgery." Procedia Computer Science 171 (2020): 2618-2625.}
see \cite{meshram20205g}

In today’s communication era, we are speedily moving from various generations of mobile communication technologies to Next Generation mobile communication. Moving from 3G, 4G and 5G, one to the next generation, it includes in-built improvements in various characteristics. The world is collecting traditional information this is huge and providing the same to every corner of the world that creates huge pressure on the core networking and backhaul resources. This paper focuses on combining 5G technologies with mobile edge computing for robotic telesurgery. A huge amount of data is required to continuously transfer over the high-speed network for this process. This process demands high network bandwidth, minimal information loss, minimal delay and real-time response for precise surgery. As of today, useful and cost-effective solutions for medical domain telerobotic surgeries are not preferred yet may be due to the performance limitations of existing communication technologies like 4G. Hence, realtime medical video transmission using 5G enabled tactile (T5ET) internet technology environment is experimented in this research, with a focus on QoS parameters like jitter control, throughput, and delay. Promising results are observed in the evaluation and discussed here which will prominently contribute to the medical domain.

\medskip
\subsubsection{Shah, Shinil K., Melissa M. Felinski, Todd D. Wilson, Kulvinder S. Bajwa, and Erik B. Wilson. "Next-Generation Surgical Robots." In Digital Surgery, pp. 401-405. Springer, Cham, 2021.}
see \cite{shah2021next}

The adoption of robotic surgery continues to increase, with over 1 million robotic-assisted surgical procedures performed worldwide (2018 estimate). Over 70 companies are developing/introducing platforms for robotic-assisted surgery in nearly every procedural speciality. In this chapter, we review concepts that are important to consider when discussing the future of robotic surgical platforms, including design of the surgeon, surgical team, and patient interfaces, integrated versus modular designs, reality augmentation, cost, and data analytics.

\medskip
\subsubsection{Ara, Jinat, Hanif Bhuiyan, Yeasin Arafat Bhuiyan, Salma Begum Bhyan, and Muhammad Ismail Bhuiyan. "AR-based Modern Healthcare: A Review." arXiv preprint arXiv:2101.06364 (2021).}
see \cite{ara2021ar}

The recent advances of Augmented Reality (AR) in healthcare have shown that technology is a significant part of the current healthcare system. In recent days, augmented reality has proposed numerous smart applications in healthcare domain including wearable access, telemedicine, remote surgery, diagnosis of medical reports, emergency medicine, etc. The aim of the developed augmented healthcare application is to improve patient care, increase efficiency, and decrease costs. This article puts on an effort to review the advances in AR-based healthcare technologies and goes to peek into the strategies that are being taken to further this branch of technology. This article explores the important services of augmented-based healthcare solutions and throws light on recently invented ones as well as their respective platforms. It also addresses concurrent concerns and their relevant future challenges. In addition, this paper analyzes distinct AR security and privacy including security requirements and attack terminologies. Furthermore, this paper proposes a security model to minimize security risks. Augmented reality advantages in healthcare, especially for operating surgery, emergency diagnosis, and medical training is being demonstrated here thorough proper analysis. To say the least, the article illustrates a complete overview of augmented reality technology in the modern healthcare sector by demonstrating its impacts, advancements, current vulnerabilities; future challenges, and concludes with recommendations to a new direction for further research.


\medskip
\subsubsection{Troccaz, Jocelyne, Giulio Dagnino, and Guang-Zhong Yang. "Frontiers of medical robotics: from concept to systems to clinical translation." Annual review of biomedical engineering 21 (2019): 193-218.}
see \cite{troccaz2019frontiers}

Medical robotics is poised to transform all aspects of medicine—from surgical intervention to targeted therapy, rehabilitation, and hospital automation. A key area is the development of robots for minimally invasive interventions. This review provides a detailed analysis of the evolution of interventional robots and discusses how the integration of imaging, sensing, and robotics can influence the patient care pathway toward precision intervention and patient-specific treatment. It outlines how closer coupling of perception, decision, and action can lead to enhanced dexterity, greater precision, and reduced invasiveness. It provides a critical analysis of some of the key interventional robot platforms developed over the years and their relative merit and intrinsic limitations. The review also presents a future outlook for robotic interventions and emerging trends in making them easier to use, lightweight, ergonomic, and intelligent, and thus smarter, safer, and more accessible for clinical use.

\medskip
\subsubsection{Dupont, Pierre E., Bradley J. Nelson, Michael Goldfarb, Blake Hannaford, Arianna Menciassi, Marcia K. O’Malley, Nabil Simaan, Pietro Valdastri, and Guang-Zhong Yang. "A decade retrospective of medical robotics research from 2010 to 2020." Science Robotics 6, no. 60 (2021): eabi8017.}
see \cite{dupont2021decade}

Robotics is a forward-looking discipline. Attention is focused on identifying the next grand challenges. In an applied field such as medical robotics, however, it is important to plan the future based on a clear understanding of what the research community has recently accomplished and where this work stands with respect to clinical needs and commercialization. This Review article identifies and analyzes the eight key research themes in medical robotics over the past decade. These thematic areas were identified using search criteria that identified the most highly cited papers of the decade. Our goal for this Review article is to provide an accessible way for readers to quickly appreciate some of the most exciting accomplishments in medical robotics over the past decade; for this reason, we have focused only on a small number of seminal papers in each thematic area. We hope that this article serves to foster an entrepreneurial spirit in researchers to reduce the widening gap between research and translation.

\medskip
\subsubsection{Hu, Jiabing, Ying Sun, Gongfa Li, Guozhang Jiang, and Bo Tao. "Probability analysis for grasp planning facing the field of medical robotics." Measurement 141 (2019): 227-234.}
see \cite{hu2019probability}

Medical surgical robot is a fusion of medical image information matching fusion technology and robotic trajectory control technology. The medical image information matching fusion is to obtain two images of a certain range of the patient’s body by two cameras on the robot, and after matching fusion processing, an image is obtained. At present, surgical robots have been successfully applied in minimally invasive surgery such as pelvic organ prolapse, defects and other basin basement reconstruction operations. Previously, most of the robots used in medical surgery have only one arm, but with the development of robotics related fields, multi-fingered robots with binocular stereo vision become possible in completing complex minimally invasive surgery. This paper aims to promote the further integration of multi-fingered manipulator and medical image detection, focusing on the grasping probability of multi-fingered manipulator. When the three-dimensional information of the object is incomplete, the machine learning method performs better than the hard coding method in the object grasping point planning. At present, most known methods can obtain classification results but could not give the probability of this category. Aiming at the problem of grab point planning, this paper proposes a crawling planning method based on big data Gaussian process classification. In this paper, a planner based on Gaussian process classification is designed, and the hyper constant used in the Gaussian process to judge the probability of capture is calculated. Based on the determined crawling scheme, the feasibility distribution map of the grab points which obtained by the trained Gaussian process classifier is drawn in MATLAB. The results show that the trained Gaussian process classifier is biased towards the center of the object which is the point with high stability. This method can give classification results and corresponding probabilities, which represents the feasibility of grasping points.

\medskip
\subsubsection{Hsiao, Jen-Hsuan, Jen-Yuan Chang, and Chao-Min Cheng. "Soft medical robotics: clinical and biomedical applications, challenges, and future directions." Advanced Robotics 33, no. 21 (2019): 1099-1111.}
see \cite{hsiao2019soft}

Bioinspired soft robotics allow for safer clinical interactions with human patients but conventional, hard robots, which are often built with rigid materials and complex control systems, compromise tissue integrity, freedom of movement, conformability, and overall human bio-compatibility. Soft, compliant materials intrinsically reduce mechanical complexity, accommodate their usage environment, and provide great practical potential for medical device developments. Previous review papers have generally covered the topics of materials, manufacturing processes, actuator modeling and control, and current trends. Here, we focus on recent developments in soft robotic applications for the medical field including advances in cardiac devices, surgical robots, and soft rehabilitation and assistance devices. In medical applications, soft robotic devices not only expedite the evolution of minimally invasive surgery but also improve the bio-compatibility of rehabilitation and assistance devices. Here, we evaluate design requirements, mechanisms, achievements and challenges in these key areas. Of particular note, this paper concludes with a discussion on advances in 3D printing and adapting neural networks for modeling and control frameworks that have facilitated the development of faster and less expensive soft medical devices.

\medskip
\subsubsection{Bai, Long, Jianxing Yang, Xiaohong Chen, Yuanxi Sun, and Xingyu Li. "Medical robotics in bone fracture reduction surgery: a review." Sensors 19, no. 16 (2019): 3593.}
see \cite{bai2019medical}

Since the advantages of precise operation and effective reduction of radiation, robots have become one of the best choices for solving the defects of traditional fracture reduction surgery. This paper focuses on the application of robots in fracture reduction surgery, design of the mechanism, navigation technology, robotic control, interaction technology, and the bone–robot connection technology. Through literature review, the problems in current fracture reduction robot and its future development are discussed.

\medskip
\subsubsection{Taylor, Russell H., Peter Kazanzides, Gregory S. Fischer, and Nabil Simaan. "Medical robotics and computer-integrated interventional medicine." In Biomedical Information Technology, pp. 617-672. Academic Press, 2020.}
see \cite{taylor2020medical}

This chapter is concerned with medical robotics and computer-integrated interventional medicine (CIIM). Broadly speaking, CIIM systems enable a three-way partnership between humans, technology, and information to improve treatment processes in surgery and other forms of interventional medicine. We first review the architecture, basic mathematical methods, and technology found in such systems and briefly discuss some of the common safety and regulatory compliance issues associated with them. Then we provide two common and interrelated paradigms found in CIIM systems and provide a few selected examples of each paradigm. The first paradigm (which we call “surgical CAD/CAM”) emphasizes CIIM as a closed-loop process consisting of (1) constructing a patient-specific model and interventional plan; (2) registering the model and plan to the patient; (3) using technology to assist in carrying out the plan; and (4) assessing the result. The second paradigm, which we refer to as “surgical assistance,” emphasizes the interactive nature of CIIM systems, in which surgical decisions are made in the operating room.

\medskip
\subsubsection{Liu, Jun, Gurpreet Singh, Subhi Al'Aref, Benjamin Lee, Olachi Oleru, James K. Min, Simon Dunham, Mert R. Sabuncu, and Bobak Mosadegh. "Image registration in medical robotics and intelligent systems: fundamentals and applications." Advanced Intelligent Systems 1, no. 6 (2019): 1900048.}
see \cite{liu2019image}

Medical image registration, by transforming two or more sets of imaging data into one coordinate system, plays a central role in medical robotics and intelligent systems from diagnostics and surgical planning to real-time guidance and postprocedural assessment. Recent advances in medical image registration have made a significant impact in orthopedic, neurological, cardiovascular, and oncological applications.The recent literature in medical image registration is reviewed, providing a discussion of their fundamentals and applications. Within each section, the registration techniques are introduced, classifying each method based on their working mechanisms, and discussing their benefits and limitations are discussed. Recently, machine learning has had an important impact on the field of image registration, yielding novel methods and unprecedented speed. The validation of registration methods, however, remains a challenge due to the lack of reliable ground truth. Medical image registration will continue to make significant impacts in the area of advanced medical imaging, as the fusion/combination of multimodal images and advanced visualization technology become more widespread.

\medskip
\subsubsection{Buettner, Ricardo, Alena Renner, and Anna Boos. "A systematic literature review of research in the surgical field of medical robotics." In 2020 IEEE 44th Annual Computers, Software, and Applications Conference (COMPSAC), pp. 517-522. IEEE, 2020.}
see \cite{buettner2020systematic}

Due to the growing demand for efficient and precise surgical options, there has been a push in the field of development of medical robot systems in recent years. This paper provides an overview of the current status and development of medical robots in surgery through a systematic literature review of research. The classification is made into minimally invasive and non-invasive robotic assistance. An introduction and definition of medical robots is provided. The advantages and disadvantages of the applications are highlighted. A summary and insight into future work is presented.

\medskip
\subsubsection{Mois, George, and Jenay M. Beer. "The Role of Healthcare Robotics in Providing Support to Older Adults: a Socio-ecological Perspective." Current Geriatrics Reports 9, no. 2 (2020): 82-89.}
see \cite{mois2020role}

Purpose of Review

In this review, we provide an overview of how healthcare robotics can facilitate healthy aging, with an emphasis on physical, cognitive, and social supports. We next provide a synthesis of future challenges and considerations in the development and application of healthcare robots. We organize these considerations using a socio-ecological perspective and discuss considerations at the individual, care partner, community healthcare, and healthcare policy levels.

Recent Findings

Older adults are the fastest growing segment of the US population. Age-related changes and challenges can present difficulties, for older adults want to age healthily and maintain independence. Technology, specifically healthcare robots, has potential to provide health supports to older adults. These supports span widely across the physical, cognitive, and social aspects of healthy aging.

Summary

Our review suggests that while healthcare robotics has potential to revolutionize the way in which older adults manage their health, there are many challenges such as clinical effectiveness, technology acceptance, health informatics, and healthcare policy and ethics. Addressing these challenges at all levels of the healthcare system will help ensure that healthcare robotics promote healthy aging and are applied safely, effectively, and reliably.

\medskip
\subsubsection{Pichetworakoon, Arachamon, Nitchanand Kooptarnond, and Sutthichai Ngamchuensuwan. "Economic and Legal on The Deploying of Medical and Healthcare Robotics: Case Study on a Comparison of the European Union (EU), South Africa, and Thailand." The Journal of Law, Public Administration and Social Science. School of Law Chiang Rai Rajabhat University 5, no. 2 (2021): 21-43.}
see \cite{pichetworakoon2021economic}

Europe is well placed to benefit from the potential of Artificial Intelligence (AI). It produces industrial and professional service robots for healthcare and plays an important role in developing and using software applications for companies. Surprisingly, the African Region and Thailand, neither of which is traditionally associated with robotics technology, are making good progress in the use and development of robots in the field of medical service and healthcare by promoting investment in robotics innovation. However, there are still several unclear aspects to be addressed by policymakers in these two regions. Although medical robots have shown great potential in these two areas by contributing to various healing processes, many limitations to the application of the technology as such have emerged in terms of economic policy, legal frameworks, the risk to privacy, and moral responsibility. Apparently, the main barrier holding back these two communities from being the next generation of automotive developers in medical robots is that legislative and investment policies governing robot activities are produced and enforced by different organizations separately. This effectively discourages not only management policy but also related action to support robotics innovation. Even though experts are creating increasingly advanced robot technology, regulation of its development is still lagging behind. This article will inform the social sciences, ethics, law, and market policy to find a solution where robots and humans can work side by side, with an emphasize on the application of legal and economic regulations relating to this growth in automation to encourage the status of the robots, bringing them to the forefront of the socio-scientific platform by applying documentary and action research methodology. In order to achieve this goal, economic policymakers and legal regulators have to engage in these agendas together with producers to establish how law and market policy should react to medical robots appropriately.

\medskip
\subsubsection{Wehde, Mark. "Healthcare 4.0." IEEE Engineering Management Review 47, no. 3 (2019): 24-28.}
see \cite{wehde2019healthcare}

Healthcare is shifting from traditional hospital-centric care to a more virtual, distributed care that heavily leverages the latest technologies around artificial intelligence, deep learning, data analytics, genomics, home-based healthcare, robotics, and three-dimensional printing of tissue and implants. In the future, fundamental shifts will reshape the healthcare industry. Healthcare will be delivered as a seamless continuum of care, away from the clinic-centered point-of-care model and with a greater focus on prevention and early intervention.

\medskip
\subsubsection{Brannan, Laura. "Inference over Knowledge Representations Automatically Generated from Medical Texts with Applications in Healthcare Robotics." (2021).}
see \cite{brannan2021inference}

Many nations across the world, including the United States, face an impending shortage of trained medical professionals and personnel. The development of a robotic healthcare assistant would help alleviate this ongoing shortage in healthcare workers. For a robotic healthcare assistant to be useful, it must facilitate human-like interactions and maintain contextual understanding of its environment. In this work, we take steps toward endowing healthcare assistant robots with the ability to anticipate the equipment needs of healthcare providers without being explicitly asked. We utilize an automatically formulated knowledge representation from web-based knowledge bases paired with a traversal algorithm to achieve these objectives. Equipped with a proper knowledge base and rule-based traversal algorithm, our robot will have the ability to retrieve relevant related information given a medical condition or symptom.

\medskip
\subsubsection{Alla, Sujatha, and Pilar Pazos. "Healthcare Robotics: Key Factors that Impact Robot Adoption in Healthcare." In IIE Annual Conference. Proceedings, pp. 1121-1126. Institute of Industrial and Systems Engineers (IISE), 2019.}
see \cite{alla2019healthcare}

In the current dynamic business environment, healthcare organizations are focused on improving patient satisfaction, performance, and efficiency. The healthcare industry is considered a complex system that is highly reliant of new technologies to support clinical as well as business processes. Robotics is one of such technologies that is considered to have the potential to increase efficiency in a wide range of clinical services. Although the use of robotics in healthcare is at the early stages of adoption, some studies have shown the capacity of this technology to improve precision, accessibility through less invasive procedures, and reduction of human error during complex surgeries. Additionally, experts have anticipated an increase in the use of robots for tasks that require physical strength, and also tasks that are repetitive, unsafe or that might be contagious. Several studies reported cost savings as a result of using clinical robots. Although robotics shows promise to reduce healthcare cost, the current hospital systems are still not using them in large scale. The application of robotics in healthcare constitutes a problem of integration of new technology in an existing and highly regulated complex system. This paper will present a classification of robots based on their role in a healthcare facility and identify the key factors that affect the integration of robotics in healthcare by applying a prior technology integration framework. 

\medskip
\subsubsection{Johanson, Deborah L., Ho Seok Ahn, and Elizabeth Broadbent. "Improving Interactions with Healthcare Robots: A Review of Communication Behaviours in Social and Healthcare Contexts." International Journal of Social Robotics (2020): 1-16.}
see \cite{johanson2020improving}

A growing shortfall exists between the number of older individuals who require healthcare support and the number of qualified healthcare professionals who can provide this. Robots offer the potential to provide healthcare support to patients both at home and in healthcare settings. However, in order for robots to be successfully implemented in these environments, they need to behave in ways that are appropriate and acceptable to human users. One way to identify appropriate social behaviours for healthcare robots is to model their behaviour on interactions between healthcare professionals and patients. This literature review aimed to inform healthcare robotics research by highlighting communication behaviours that are important within the context of healthcare. The review focussed on relevant research in human clinical interactions, followed by a review of similar factors in social robotics research. Three databases were searched for terms relating to healthcare professional communication behaviours associated with patient outcomes. The results identified key communication behaviours that can convey clinical empathy, including humour, self-disclosure, facial expressions, eye gaze, body posture, and gestures. A further search was conducted to identify research examining these key behaviours within the context of social and healthcare robotics. Research into these factors in human–robot interaction in healthcare is limited to date, and this review provides a useful guide for future research.

\medskip
\subsubsection{Radic, Marija, Agnes Vosen, and Birgit Graf. "Use of robotics in the German healthcare sector." In International Conference on Social Robotics, pp. 434-442. Springer, Cham, 2019.}
see \cite{radic2019use}

This page seeks to help scientists and the wider community better think that makes a robot pleasant by giving an overview of healthcare robot ideas, laboratory testing, and applications. When healthcare robots are utilized appropriately for their structure and functions, they show their capabilities. Companions for the elderly and others with cognitive impairments, robots in educational settings, and cognitive and behavioral enhancement technology are just a few examples. While the robots shown in films and literature remain futuristic, science fiction has inspired everybody to envision a world in which robotics help us in every aspect of our everyday lives. While we have a long way to go before robots are ubiquitous in our social spaces, significant advances in healthcare robotics technology, supported by social sciences, are bringing us closer.

\medskip
\subsubsection{Bartosiak, Marcin, Gianni Bonelli, Lorenzo Stefano Maffioli, Ugo Palaoro, Francesco Dentali, Giovanni Poggialini, Federica Pagliarin, Stefano Denicolai, and Pietro Previtali. "Advanced Robotics as a Support in Healthcare Organizational Response. A COVID-19 Pandemic case." In Healthcare Management Forum, p. 08404704211042467. Sage CA: Los Angeles, CA: SAGE Publications, 2021.}
see \cite{bartosiak2021advanced}

The use of robotics is becoming widespread in healthcare. However, little is known about how robotics can affect the relationship with patients in epidemic emergency response or how it impacts clinicians in their organization and work. As a hospital responding to the consequences of the COVID-19 pandemic “ASST dei Sette Laghi” (A7L) in Varese, Italy, had to react quickly to protect its staff from infection while coping with high budgetary pressure as prices of Personal Protection Equipment (PPE) increased rapidly. In response, it introduced six semi-autonomous robots to mediate interactions between staff and patients. Thanks to the cooperation of multiple departments, A7L implemented the solution in less than 10 weeks. It reduced risks to staff and outlay for PPE. However, the characteristics of the robots affected their perception by healthcare staff. This case study reviews critical issues faced by A7L in introducing these devices and recommendations for the path forward.

\medskip
\subsubsection{Khan, Arshia, and Yumna Anwar. "Robots in healthcare: A survey." In Science and Information Conference, pp. 280-292. Springer, Cham, 2019.}
see \cite{khan2019robots}

Advances in robotic technology is stimulating growth in new treatment mechanism by enhancing patient outcomes and helping reduce healthcare costs, while providing alternate care apparatus. Provision of care by assistive therapeutic robots has increasingly grown in the past decade. Although the healthcare industry has been lagging in the use of assistive robots; the use of assistive robots in the manufacturing industry has been a norm for a long time. The vulnerable population of patients with illnesses, cognition challenges, and disabilities are some of the causes for the delay in the use of assistive therapeutic robots in healthcare. In this paper we explore the various types of assistive robots and their use in the healthcare industry.

\medskip
\subsubsection{Ikeda, Yoko, and Michiko Iizuka. Global Rulemaking Strategy for Implementing Emerging Innovation: Case of Medical/Healthcare Robot, HAL by Cyberdyne (Japanese). Research Institute of Economy, Trade and Industry (RIETI), 2019.}
see \cite{ikeda2019global}

Robots have been put to use in many fields mostly for automation or areas where a great degree of precision is required. Robots can be of huge assistance in medical field too, as they can relieve the patient or the medical personnel from routine and mundane tasks, which may sometime be very crucial and may need to be performed with utmost care, accuracy and precision. The use of robotics is already there in healthcare, but it's not main-stream yet and it would take some time for that to become a reality. The main goal of this research paper would be to shed some light on the same. I have proposed some ideas on how robotics can be used in some niche in healthcare, and how it can be made easy to spread and implement on the ground level. Focus on the need of robotics in healthcare, along with their added advantages in the quality of healthcare and the savings in long time costs would be there. With this, the future of healthcare i.e. Telemedicine would become a reality and it would be a lot easier and cheaper for people to get access to quality healthcare, anywhere in the world with physically attending the hospital.

\medskip
\subsubsection{Sarker, Sujan, Lafifa Jamal, Syeda Faiza Ahmed, and Niloy Irtisam. "Robotics and artificial intelligence in healthcare during COVID-19 pandemic: A systematic review." Robotics and autonomous systems (2021): 103902.}
see \cite{sarker2021robotics}

The outbreak of the COVID-19 pandemic is unarguably the biggest catastrophe of the 21st century, probably the most significant global crisis after the second world war. The rapid spreading capability of the virus has compelled the world population to maintain strict preventive measures. The outrage of the virus has rampaged through the healthcare sector tremendously. This pandemic created a huge demand for necessary healthcare equipment, medicines along with the requirement for advanced robotics and artificial intelligence-based applications. The intelligent robot systems have great potential to render service in diagnosis, risk assessment, monitoring, telehealthcare, disinfection, and several other operations during this pandemic which has helped reduce the workload of the frontline workers remarkably. The long-awaited vaccine discovery of this deadly virus has also been greatly accelerated with AI-empowered tools. In addition to that, many robotics and Robotics Process Automation platforms have substantially facilitated the distribution of the vaccine in many arrangements pertaining to it. These forefront technologies have also aided in giving comfort to the people dealing with less addressed mental health complicacies. This paper investigates the use of robotics and artificial intelligence-based technologies and their applications in healthcare to fight against the COVID-19 pandemic. A systematic search following the Preferred Reporting Items for Systematic Reviews and Meta-Analyses (PRISMA) method is conducted to accumulate such literature, and an extensive review on 147 selected records is performed.


\medskip
\subsubsection{Alotaibi, Meshal, and Mohammad Yamin. "Role of robots in healthcare management." In 2019 6th International Conference on Computing for Sustainable Global Development (INDIACom), pp. 1311-1314. IEEE, 2019.}
see \cite{alotaibi2019role}

Robots in medical science and healthcare are playing significant roles by doing procedures and other tasks which earlier were being done by humans. Robots are assisting health patients, administrators, healthcare systems and entities in many ways to improve the health and well-being of the people. Robots have simplified many complex surgical procedures which earlier were very obscure. Although there is a lot of literature on various aspects of healthcare, but there still is a scope for having a detailed review of how robots can help various entities of the healthcare management. This paper intends to fill that gap and provides an exposition of the role of robots in healthcare and health procedures, especially the surgical procedures and patient assistance. Our review exhibits various aspects of hospital and healthcare management and patient administration.

\medskip
\subsubsection{Aggarwal, Shivangi, Deepa Gupta, and Sonia Saini. "A literature survey on robotics in healthcare." In 2019 4th International Conference on Information Systems and Computer Networks (ISCON), pp. 55-58. IEEE, 2019.}
see \cite{aggarwal2019literature}

In this paper we will become familiar with the utilization of various kinds of robots in therapeutic and social insurance part. How robots are being utilized in therapeutic division, their applications, favorable circumstances and impediments. Therapeutic robots much of the time utilized in Neurosurgery, Laparoscopic, Orthoprodic basically. Robots are being utilized in tele medical procedure too. Robots have been giving invasiness in determination. This paper will talk about them quickly. Robots are utilized for recovery purposes. The robots in the medicinal services industry largy affect the social insurance division by expanding labor and substantially more. Restorative mechanical autonomy has become quicker in recent years and growing always. Restorative mechanical technology is expanding in the field of Nanotechnology, Orthoprodics and different medical procedures. The presence of apply autonomy is changing the businesses over the globe. Mechanical autonomy is expanding step by step and covering functionalities, for example, nanotechnology, man-made reasoning to adjust the human services and different areas. The objective of the paper was to comprehend the job of mechanical technology in medicinal services area and its future difficulties and the advantages of the apply autonomy in human services industry. The utilization of robots in wellbeing area gives an energizing just as a pivotal chance to help an enormous number of individuals.

\medskip
\subsubsection{Puaschunder, Julia M. "The Potential for Artificial Intelligence in Healthcare." Available at SSRN 3525037 (2020).}
see \cite{puaschunder2020potential}

Artificial Intelligence (AI), Robotics and Big Data revolutionized the world and opened unprecedented opportunities and potentials in healthcare. No other scientific field grants as much hope in the determination of life and death and fastest-pace innovation potential with economically highest profit margin prospects as does medical care.

An expert survey conducted in November 2019 identified big data-driven knowledge generation and tailored personal medical care but also efficiency, precision and better quality work as most beneficial advancements of AI, robotics and big data in the healthcare sector. Decentralized preventive healthcare and telemedicine open access to personalized, affordable healthcare.

Technical advancements and big data insights – at the same time – increase costs for a whole-roundedly healthy lifestyle. Particularly in Western Europe, the currently tipping demographic pyramid coupled with obstacles to integrate migrants long-term to rejuvenate the population and boost economic output impose challenges for policy makers and insurance practitioners alike. Studies in the US found that 70\% of all health related costs are accrued during the last few weeks of peoples’ living. In Austria, healthcare cost are expected to double in this decade. This predicament of rising costs of an aging Western world population, raises questions such as – Should we decrease the access to the best quality medical care of Austrian in order to maintain the pursuit of a mandate of medical care for all – or should we allow a different-tiered class-based medical system, in which money determines who can afford excellent healthcare? There must be a better solution for a country like Austria in the heart of the European continent that may stem from a Moving Forward thinking community as we all represent together today.

\medskip
\subsubsection{Mohanty, Kajol, S. Subiksha, S. Kirthika, B. H. Sujal, Sumathi Sokkanarayanan, Panjavarnam Bose, and Mithileysh Sathiyanarayanan. "Opportunities of Adopting AI-Powered Robotics to Tackle COVID-19." In 2021 International Conference on COMmunication Systems \& NETworkS (COMSNETS), pp. 703-708. IEEE, 2021.}
see \cite{mohanty2021opportunities}

The coronavirus disease (COVID-19) pandemic has made a dire requirement for traditional and disruptive technologies to react to the flare-up across health and wellbeing areas, and technologies such as AI and robotics have been recognized as promising ways to tackle the current challenges. The COVID-19 pandemic has exhibited the solid capability of different advanced technologies that have been tried during the emergency. However, acceptability and adoptability of the latest technologies may face serious challenges due to potential conflicts with users' cultural, moral, and religious backgrounds. This paper discusses the current opportunities and challenges with respect to artificial intelligence (AI) powered robots to battle COVID-19. To diminish the danger of contamination and infection, the opportunities must be utilized during this pandemic for a better future. More deliberate measures ought to be executed to guarantee that future robotic health initiatives will have a greater impact on the pandemic and meet the most key needs to facilitate the life of individuals who are at the forefront of the crisis.

\medskip
\subsubsection{Fosch Villaronga, Eduard, and Hadassah Drukarch. "On healthcare robots." On healthcare robots (2021).}
see \cite{fosch2021healthcare}

The rise of healthcare robotics

Robotics  have  increased  productivity  and  resource  efficiency  in  the  industrial  and  retail sectors, and now there is an emerging interest in realizing a comparable transformation in healthcare.  Robotics  and  artificial  intelligence  (AI)  are  some  of  the  latest  promising technologies  expected  to  increase  the  quality  and  safety  of  care  while  simultaneously restraining expenditure and, recently, reducing human contact too. Healthcare robots are likely to be deployed at an unprecedented rate due to their reduced cost and increasing capabilities such  as  carrying  out  medical  interventions,  supporting  biomedical  research  and clinical practice, conducting therapy with children, or keeping the elderly company.

The lack of healthcare robot policy

Although healthcare is a remarkably sensitive domain of application, and systems that exert direct control over the world can cause harm in a way that humans cannot necessarily correct or oversee, it is still unclear whether and how healthcare robots are currently regulated or should be regulated. Existing regulations are primarily unprepared to provide guidance for such a rapidly evolving field and accommodate devices that rely on machine learning and AI. Moreover, the field of healthcare robotics is very rich and extensive, butit is still very much scattered and unclear in terms of definitions, medical and technical classifications, product characteristics, purpose, and intended use. As a result, these devices often navigate between the medical device regulation or other non-medical norms, such as the ISO personal care standard. Before regulating the field of healthcare robots, it is therefore essential to map the major state-of-the-art developments in healthcare robotics, their capabilities and applications, and the challenges we face as a result of their integration within the healthcare environment.

Our contribution to the policy making debate on healthcare robots and AI technologies

This contribution fills in this gap and lack of clarity currently experienced within healthcare robotics and its governance by providing a structured overview of and further elaboration on the main categories now established, their intended purpose, use, and main characteristics. We explicitly focus on surgical, assistive, and service robots to rightfully match the definition of healthcare as the organized provision of medical care to individuals, including efforts to maintain, treat, or restore physical, mental, or emotional well-being. We complement these findings with policy recommendations to help policymakers unravel an optimal regulatory framing for healthcare robot technologies.

\medskip
\subsubsection{Su, Yun-Hsuan, Adnan Munawar, Anton Deguet, Andrew Lewis, Kyle Lindgren, Yangming Li, Russell H. Taylor, Gregory S. Fischer, Blake Hannaford, and Peter Kazanzides. "Collaborative Robotics Toolkit (CRTK): Open Software Framework for Surgical Robotics Research." In 2020 Fourth IEEE International Conference on Robotic Computing (IRC), pp. 48-55. IEEE, 2020.}
see \cite{su2020collaborative}

Robot-assisted minimally invasive surgery has made a substantial impact in operating rooms over the past few decades with their high dexterity, small tool size, and impact on adoption of minimally invasive techniques. In recent years, intelligence and different levels of surgical robot autonomy have emerged thanks to the medical robotics endeavors at numerous academic institutions and leading surgical robot companies. To accelerate interaction within the research community and prevent repeated development, we propose the Collaborative Robotics Toolkit (CRTK), a common API for the RAVEN-II and da Vinci Research Kit (dVRK) - two open surgical robot platforms installed at more than 40 institutions worldwide. CRTK has broadened to include other robots and devices, including simulated robotic systems and industrial robots. This common API is a community software infrastructure for research and education in cutting edge human-robot collaborative areas such as semi-autonomous teleoperation and medical robotics. This paper presents the concepts, design details and the integration of CRTK with physical robot systems and simulation platforms.

\medskip
\subsubsection{Javaid, Mohd, Abid Haleem, Abhishek Vaish, Raju Vaishya, and Karthikeyan P. Iyengar. "Robotics applications in COVID-19: A review." Journal of Industrial Integration and Management 5, no. 04 (2020): 441-451.}
see \cite{javaid2020robotics}

The COVID-19 outbreak has resulted in the manufacturing and service sectors being badly hit globally. Since there are no vaccines or any proven medical treatment available, there is an urgent need to take necessary steps to prevent the spread of this virus. As the virus spreads with human-to-human interaction, lockdown has been declared in many countries, and the public is advised to observe social distancing strictly. Robots can undertake human-like activities and can be gainfully programmed to replace some of the human interactions. Through this paper, we identify and propose the introduction of robots to take up this challenge in the fight against the COVID-19 pandemic. We did a comprehensive review of the literature to identify robots’ possible applications in the management of epidemics and pandemics of this nature. We have reviewed the available literature through the search engines of PubMed, SCOPUS, Google Scholar, and Research Gate. A comprehensive review of the literature identified different types of robots being used in the medical field. We could find several vital applications of robots in the management of the COVID-19 pandemic. No doubt technology comes with a cost. In this paper, we identified how different types of robots are used gainfully to deliver medicine, food, and other essential items to COVID-19 patients who are under quarantine. Therefore, there is extensive scope for customising robots to undertake hazardous and repetitive jobs with precision and reliability.

\medskip
\subsubsection{Guntur, Sitaramanjaneya Reddy, Rajani Reddy Gorrepati, and Vijaya R. Dirisala. "Robotics in healthcare: an internet of medical robotic things (IoMRT) perspective." In Machine learning in bio-signal analysis and diagnostic imaging, pp. 293-318. Academic Press, 2019.}
see \cite{guntur2019robotics}

Robotics is one of the most advanced and emerging technologies in the field of medicine. Electronic sensors incorporated with combination of control into mechanical systems greatly enhance the performance and flexibility of systems. The robotic technology used for movement of arms were not accurate and unable to send the exact sensory feedback, exact movement, and positioning. With the advances in hardware, software, and control programming systems, extensive automation is being utilized to operate with more degree of freedom than humans under an extensive array of conditions. Currently, robotic innovations are introduced in numerous areas that specifically influence the understanding and consideration of patient care. Robotics technology in medicine is the prime focus of the healthcare services in ICUs, general rooms, and surgery room which reduces risks for patients, doctors and it is also utilized in laboratories to collect the samples followed by transportation of samples if required, analyzing, and preserving them for long-term storage. The healthcare services provided by robotics become complex and critical pertaining to sharing of information, data communication and distribution of the sensors data. The Internet of Medical Robotics Things (IoMRT) approaches incorporated robots as a “thing” and buildup connections with new communication such as Li-Fi technology and information technology on web. This chapter demonstrates the overview of robotics in performing surgeries, other healthcare services and it also emphasizes on long-term benefits for human beings using robotics with IoMT-based new communication (Li-Fi) technology. Also the limitations and future challenges associated with this technology are described in detail.

\medskip
\subsubsection{Pierce, Robin, and Eduard Fosch Villaronga. "Medical robots and the right to health care: A progressive realisation." (2020).}
see \cite{pierce2020medical}

Robotic technologies have shown to have clear potential for providing innovation in treatments and treatment modalities for various diseases and disorders that cover unmet needs and are cost-efficient. However, the emergence of technology that promises to improve health outcomes raises the question regarding the extent to which it should be incorporated, how, made available to whom, and on what basis. Since countries usually have limited resources to favour access to state-of-the-art technologies and develop strategies to realize the right to health progressively, in this article, we investigate whether the right to health, particularly the core obligations specified under this right, helps implement medical robots.

\medskip
\subsubsection{Amir Hossein, Molkizadeh, Rahim Baghban, Somayeh Rahmanian, Saeed Bayyenat, and Mohammad Ali Kiani. "Telemedicine: An Essential Requirement for the Health Care Providers, with Emphasis on Legal Aspects." International Journal of Pediatrics 8, no. 9 (2020): 12131-12142.}
see \cite{amir2020telemedicine}

Telemedicine is the use of telecommunication and information technologies in order to provide clinical health care at a distance. These technologies allow communications between patient and medical staff with convenience as well as the transmission of medical, imaging and health informatics data from one site to another. It is also used to save lives in critical care and emergency situations. Although telemedicine systems have many advantages, including the distribution of high quality medical services to remote areas, failure to comply with infrastructure will reduce the efficiency and quality of their services. Issues such as building the infrastructure of the medical information industry, including the legal infrastructure, and thus providing a suitable platform for the legal and ethical issues of Telemedicine, as well as obtaining the necessary permits and requirements, will play an important role in the successful implementation of a Telemedicine system. The purpose of this study was to become more familiar with the field of Telemedicine and its services, as well as to review some legal issues in the field of e-health. Telemedicine is not able to solve the problems of the health and social systems, but the problems of the health and social systems cannot be solved without Telemedicine.


%\medskip
%\subsubsection{}
%see \cite{}




\bibliographystyle{IEEEtran}
\bibliography{lit}

\end{document}
